% use to waste space:
% \documentclass[12pt,a4paper]{article}

% if you have this style and like it.
%\documentclass{acmsiggraph}
%\documentclass[review]{acmsiggraph}      % review
%\documentclass[widereview]{acmsiggraph}  % wide-spaced review
%\documentclass[preprint]{acmsiggraph}    % preprint

% define a \comment{this is a comment which can have linebreaks in it}
\newcommand{\comment}[1]{}
% \newcommand{\todo}[1]{\marginpar{\bf{#1}}}
\newcommand{\todo}[1]{{\color{red}\bf{TODO: #1}}}

\usepackage{amsthm}
\usepackage{amsfonts}
\usepackage{mathptmx}
\usepackage{stmaryrd}
\usepackage{wasysym} %?
\usepackage{esvect}
\usepackage{graphicx}
\usepackage{color}
\definecolor{rot}{RGB}{165,30,55} %rote Farbe
\graphicspath{{./images/}}
\usepackage{parskip}
\usepackage{dsfont}
\usepackage{pxfonts}
\usepackage{tikz}

% comment these two lines out if you don't want minion/myriad fonts.
%\usepackage[minionint,mathlf]{MinionPro}
% \renewcommand{\sfdefault}{Myriad-LF}
%\usepackage{Myriad}
\usepackage{fontspec}
\setmonofont{DejaVu Sans Mono}

% no page number on float pages, fixes problems with overlarge diagrams.
\usepackage{fancyhdr}
\pagestyle{fancy}
%\lhead{}
%\chead{}
%\rhead{}
%\lfoot{}
\fancyhf{}
\fancyhead[EL]{\nouppercase{\leftmark}}
\fancyhead[OR]{\nouppercase{\rightmark}}
\cfoot{}
%\fancyfoot[EL]{\iffloatpage{}{\thepage}}
%\fancyfoot[OR]{\iffloatpage{}{\thepage}}
\fancyfoot[EL]{\thepage}
\fancyfoot[OR]{\thepage}
\renewcommand{\headrulewidth}{0pt}
\renewcommand{\footrulewidth}{0pt}

%\usepackage{natbib}		% textual referencing
%\usepackage[numbers,super]{natbib}	% nice superscripts
%\bibliographystyle{chicago}	% shitty
\bibliographystyle{alpha}	% abbr names and year in \cite
%\bibliographystyle{agsm}	% australian, need natbib
%\bibliographystyle{kluwer}	% need natbib
%\bibliographystyle{apalike}	% lengthly
%\bibliographystyle{abbrv}	% minimal?

% use for german line breaking:
%\usepackage[ngerman]{babel}
\usepackage[T1]{fontenc}

% avoid us-style text color destruction:
\frenchspacing
\usepackage{microtype}

% have a nice framebox with border directly around the image:
\fboxsep 0pt
\newcommand{\fimg}[2]{\fbox{\includegraphics[width=#1]{#2}}}

\newtheorem{theorem}{Theorem}
\newtheorem{definition}{Definition}

\usepackage{listings}
\lstset{basicstyle=\scriptsize, mathescape}
\usepackage{minted}
\usepackage[boxruled]{algorithm2e}
%\usepackage{hyperref}
\usepackage{url}
\usepackage{subfig}

\def\code#1{{\tt{#1}}}

\usepackage{bussproofs}
\usepackage{xcolor}
\usepackage{lscape}

\newcommand{\id}[1]{\text{id}_{#1}}
\newcommand{\rec}{\text{rec}}
\newcommand{\recT}[2]{\rec\;#1\text{ to } #2\text{ where}}
\newcommand{\corec}{\text{corec}}
\newcommand{\TyCtx}{\enskip\textbf{TyCtx}}
\newcommand{\ParCtx}{\enskip\textbf{ParCtx}}
\newcommand{\Ctx}{\enskip\textbf{Ctx}}
\newcommand{\D}{\AxiomC{$\mathcal{D}$}\noLine}
\newcommand{\Di}[1]{\AxiomC{$\mathcal{D}_{#1}$}\noLine}
\newcommand{\topI}[1]{\RightLabel{\textbf{($\top$-I)}}\UnaryInfC{#1}}
\newcommand{\TyVarI}[1]{\RightLabel{\textbf{TyVar-I}}\BinaryInfC{#1}}
\newcommand{\TyVarWeak}[1]{\RightLabel{\textbf{(TyVar-Weak)}}\BinaryInfC{#1}}
\newcommand{\TyWeak}[1]{\RightLabel{\textbf{(Ty-Weak)}}\BinaryInfC{#1}}
\newcommand{\TyInstLabel}{\RightLabel{\textbf{(Ty-Inst)}}}
\newcommand{\TyInst}[1]{\TyInstLabel\BinaryInfC{#1}}
\newcommand{\TyInstTrinary}[1]{\TyInstLabel\TrinaryInfC{#1}}
\newcommand{\ParamAbstr}[1]{\RightLabel{\textbf{(Param-Abstr)}}\UnaryInfC{#1}}
\newcommand{\FPTy}{\RightLabel{\textbf{(FP-Ty)}}}
\newcommand{\Inst}[1]{\RightLabel{\textbf{(Inst)}}\BinaryInfC{#1}}
\newcommand{\Conv}[1]{\RightLabel{\textbf{(Conv)}}\BinaryInfC{#1}}
\newcommand{\Proj}[1]{\RightLabel{\textbf{(Proj)}}\UnaryInfC{#1}}
\newcommand{\TermWeak}[1]{\RightLabel{\textbf{(Term-Weak)}}\BinaryInfC{#1}}
\newcommand{\IndILabel}{\RightLabel{\textbf{(Ind-I)}}}
\newcommand{\IndI}[1]{\IndILabel\UnaryInfC{#1}}
\newcommand{\IndIBinary}[1]{\IndILabel\BinaryInfC{#1}}
\newcommand{\CoindE}[1]{\RightLabel{\textbf{(Coind-E)}}\UnaryInfC{#1}}
\newcommand{\IndE}{\RightLabel{\textbf{(Ind-E)}}}
\newcommand{\CoindI}{\RightLabel{\textbf{(Coind-I)}}}
\newcommand{\rat}{\rightarrowtriangle}
\newcommand{\graybox}[1]{\colorbox{gray}{#1}}
\newcommand{\Unit}{\text{Unit}}
\newcommand{\Maybe}{\text{Maybe}}
\newcommand{\Just}{\text{Just}}
\newcommand{\Nothing}{\text{Nothing}}
\newcommand{\nothing}{\text{nothing}}
\newcommand{\Conat}{\text{Conat}}
\newcommand*\circled[1]{\tikz[baseline=(char.base)]{
    \node[shape=circle,draw,inner sep=1pt] (char) {#1};}}

\newenvironment{scprooftree}[1]%
 {\gdef\scalefactor{#1}\begin{center}\proofSkipAmount \leavevmode}%
 {\scalebox{\scalefactor}{\DisplayProof}\proofSkipAmount \end{center} }
\newenvironment{changemargin}[2]{%
\begin{list}{}{%
\setlength{\topsep}{0pt}%
\setlength{\leftmargin}{#1}%
\setlength{\rightmargin}{#2}%
\setlength{\listparindent}{\parindent}%
\setlength{\itemindent}{\parindent}%
\setlength{\parsep}{\parskip}%
}%
\item[]}{\end{list}}