% Created 2020-11-30 Mon 17:34
% Intended LaTeX compiler: pdflatex
\documentclass[a4paper,cleardoubleempty,BCOR1cm]{scrbook}
\usepackage[utf8]{inputenc}
\usepackage[T1]{fontenc}
\usepackage{graphicx}
\usepackage{grffile}
\usepackage{longtable}
\usepackage{wrapfig}
\usepackage{rotating}
\usepackage[normalem]{ulem}
\usepackage{amsmath}
\usepackage{textcomp}
\usepackage{amssymb}
\usepackage{capt-of}
\usepackage{hyperref}
% use to waste space:
% \documentclass[12pt,a4paper]{article}

% if you have this style and like it.
%\documentclass{acmsiggraph}
%\documentclass[review]{acmsiggraph}      % review
%\documentclass[widereview]{acmsiggraph}  % wide-spaced review
%\documentclass[preprint]{acmsiggraph}    % preprint

% define a \comment{this is a comment which can have linebreaks in it}
\newcommand{\comment}[1]{}
% \newcommand{\todo}[1]{\marginpar{\bf{#1}}}
\newcommand{\todo}[1]{{\color{red}\bf{TODO: #1}}}

\usepackage{amsthm}
\usepackage{mathptmx}
\usepackage{stmaryrd}
\usepackage{wasysym} %?
\usepackage{esvect}
\usepackage{graphicx}
\usepackage{color}
\definecolor{rot}{RGB}{165,30,55} %rote Farbe
\graphicspath{{./images/}}
\usepackage{parskip}
\usepackage{dsfont}
\usepackage{pxfonts}

% comment these two lines out if you don't want minion/myriad fonts.
%\usepackage[minionint,mathlf]{MinionPro}
% \renewcommand{\sfdefault}{Myriad-LF}
%\usepackage{Myriad}
\usepackage{fontspec}
\setmonofont{DejaVu Sans Mono}

% no page number on float pages, fixes problems with overlarge diagrams.
\usepackage{fancyhdr}
\pagestyle{fancy}
%\lhead{}
%\chead{}
%\rhead{}
%\lfoot{}
\fancyhf{}
\fancyhead[EL]{\nouppercase{\leftmark}}
\fancyhead[OR]{\nouppercase{\rightmark}}
\cfoot{}
%\fancyfoot[EL]{\iffloatpage{}{\thepage}}
%\fancyfoot[OR]{\iffloatpage{}{\thepage}}
\fancyfoot[EL]{\thepage}
\fancyfoot[OR]{\thepage}
\renewcommand{\headrulewidth}{0pt}
\renewcommand{\footrulewidth}{0pt}

%\usepackage{natbib}		% textual referencing
%\usepackage[numbers,super]{natbib}	% nice superscripts
%\bibliographystyle{chicago}	% shitty
\bibliographystyle{alpha}	% abbr names and year in \cite
%\bibliographystyle{agsm}	% australian, need natbib
%\bibliographystyle{kluwer}	% need natbib
%\bibliographystyle{apalike}	% lengthly
%\bibliographystyle{abbrv}	% minimal?

% use for german line breaking:
%\usepackage[ngerman]{babel}
\usepackage[T1]{fontenc}

% avoid us-style text color destruction:
\frenchspacing
\usepackage{microtype}

% have a nice framebox with border directly around the image:
\fboxsep 0pt
\newcommand{\fimg}[2]{\fbox{\includegraphics[width=#1]{#2}}}

\newtheorem{theorem}{Theorem}
\newtheorem{definition}{Definition}

\usepackage{listings}
\lstset{numbers=left, numberstyle=\tiny, basicstyle=\tiny, mathescape}
\usepackage{minted}
\usepackage[boxruled]{algorithm2e}
%\usepackage{hyperref}
\usepackage{url}
\usepackage{subfig}

\def\code#1{{\tt{#1}}}

\usepackage{bussproofs}
\usepackage{xcolor}
\usepackage{lscape}

\newcommand{\id}[1]{\text{id}_{#1}}
\newcommand{\rec}{\text{rec}}
\newcommand{\corec}{\text{corec}}
\newcommand{\TyCtx}{\enskip\textbf{TyCtx}}
\newcommand{\Ctx}{\enskip\textbf{Ctx}}
\newcommand{\D}{\AxiomC{$\mathcal{D}$}\noLine}
\newcommand{\Di}[1]{\AxiomC{$\mathcal{D}_{#1}$}\noLine}
\newcommand{\topI}[1]{\RightLabel{\textbf{($\top$-I)}}\UnaryInfC{#1}}
\newcommand{\TyVarI}[1]{\RightLabel{\textbf{TyVar-I}}\BinaryInfC{#1}}
\newcommand{\TyVarWeak}[1]{\RightLabel{\textbf{(TyVar-Weak)}}\BinaryInfC{#1}}
\newcommand{\TyWeak}[1]{\RightLabel{\textbf{(Ty-Weak)}}\BinaryInfC{#1}}
\newcommand{\TyInst}[1]{\RightLabel{\textbf{(Ty-Inst)}}\BinaryInfC{#1}}
\newcommand{\ParamAbstr}[1]{\RightLabel{\textbf{(Param-Abstr)}}\UnaryInfC{#1}}
\newcommand{\FPTy}{\RightLabel{\textbf{(FP-Ty)}}}
\newcommand{\Inst}[1]{\RightLabel{\textbf{(Inst)}}\BinaryInfC{#1}}
\newcommand{\Conv}[1]{\RightLabel{\textbf{(Conv)}}\BinaryInfC{#1}}
\newcommand{\Proj}[1]{\RightLabel{\textbf{(Proj)}}\UnaryInfC{#1}}
\newcommand{\TermWeak}[1]{\RightLabel{\textbf{(Term-Weak)}}\BinaryInfC{#1}}
\newcommand{\IndI}[1]{\RightLabel{\textbf{(Ind-I)}}\UnaryInfC{#1}}
\newcommand{\CoindE}[1]{\RightLabel{\textbf{(Coind-E)}}\UnaryInfC{#1}}
\newcommand{\IndE}{\RightLabel{\textbf{(Ind-E)}}}
\newcommand{\CoindI}{\RightLabel{\textbf{(Coind-I)}}}
\newcommand{\rat}{\rightarrowtriangle}

\newenvironment{scprooftree}[1]%
 {\gdef\scalefactor{#1}\begin{center}\proofSkipAmount \leavevmode}%
 {\scalebox{\scalefactor}{\DisplayProof}\proofSkipAmount \end{center} }
\newenvironment{changemargin}[2]{%
\begin{list}{}{%
\setlength{\topsep}{0pt}%
\setlength{\leftmargin}{#1}%
\setlength{\rightmargin}{#2}%
\setlength{\listparindent}{\parindent}%
\setlength{\itemindent}{\parindent}%
\setlength{\parsep}{\parskip}%
}%
\item[]}{\end{list}}
\author{Florian Engel}
\date{\today}
\title{Implementation of Type Theory based on Dependent Inductive and Coinductive Types}
\hypersetup{
 pdfauthor={Florian Engel},
 pdftitle={Implementation of Type Theory based on Dependent Inductive and Coinductive Types},
 pdfkeywords={},
 pdfsubject={},
 pdfcreator={Emacs 26.1 (Org mode 9.1.9)}, 
 pdflang={English}}
\begin{document}

\maketitle
\begin{tabular}{lr}
% \includegraphics[width=0.5\linewidth]{logo_sw} % logo bw
 \includegraphics[width=0.5\linewidth]{UT_WBMW_Rot_4C} % logo red
 & \hspace{0.2\linewidth}
 \parbox{0.5\linewidth}{
   \large\bf\textsf{\color{rot}{Mathematisch-\\Naturwissenschaftliche\\Fakultät\\\\}}
   \hspace{-.144cm}\normalsize\textsf{\color{rot}{Programiersprachen}}
   \vspace{0.6cm}
 }
\end{tabular}

\vspace*{10ex}
Masterarbeit

{\huge\bf\textsf{Implementation of Type Theory based on dependent inductive and
    coinductive types}}

\vspace*{30ex}

Eberhard Karls Universität Tübingen\\
Mathematisch-Naturwissenschaftliche Fakultät\\
Wilhelm-Schickard-Institut für Informatik\\
Programiersprachen\\
Florian Engel,~ \verb+florian.engel@student.uni-tuebingen.de+,~ 2012

\vspace*{5ex}

\begin{tabular}{@{}l@{\hspace{2em}}l}
  Bearbeitungszeitraum:& von-bis \vspace*{5ex} \\
  Betreuer/Gutachter:& Prof. Dr. Klaus Ostermann, Universität Tübingen\\
  Zweitgutachter:& Prof. Dr. Max Mustermann, Universität Tübingen
\end{tabular}

\thispagestyle{empty}
\newpage

\chapter*{Selbstst\"andigkeitserkl\"arung}
Hiermit versichere ich, dass ich die vorliegende Masterarbeit selbst\"andig und
nur mit den angegebenen Hilfsmitteln angefertigt habe und dass alle Stellen,
die dem Wortlaut oder dem Sinne nach anderen Werken entnommen sind,
durch Angaben von Quellen als Entlehnung kenntlich gemacht worden sind.
Diese Masterarbeit wurde in gleicher oder \"ahnlicher Form in keinem anderen
Studiengang als Pr\"ufungsleistung vorgelegt.

\vspace*{8ex}
\hrule
\vspace*{2ex}
Florian Engel (Matrikelnummer 3860700), \today



\chapter*{Abstract}
  Dependent types are a useful tool to restrict types even further than types in
  strongly typed languages like Haskell. This gives us further type safety. With
  dependent types, we can also prove theorems. Coinductive types allow us to define types by
  their observations rather than by their constructors. This is useful for
  infinite types like streams. In many common dependently typed languages, like
  Coq and Agda, we can define inductive types which depend on values and
  coinductive types but not coinductive types, which depend on values.

In this work, we will first give a survey of coinductive types in Coq and Agda
languages and then implement the type theory from \cite{basold2016type}. This
type theory has both dependent inductive types and dependent coinductive
types. In this type theory, the dependent function space becomes definable.
This leads to a more symmetrical approach to coinduction in dependently typed
languages.

\setcounter{tocdepth}{2}
\tableofcontents

\chapter{Introduction}
\label{sec:orgd9226af}
In functional programming, we use functions that consume input and produce
output. These functions don't depend on external values i.e. if there is no IO
involved, they always produce the same output for the same input. For
example, if we call a function $\mathtt{or}$\;on the values $\mathtt{true}$\;and $\mathtt{false}$\;we always
get $\mathtt{true}$. This makes code more predictable.

The $\mathtt{or}$\;function should only be working on booleans. To call $\mathtt{or}$\;on strings
$\mathtt{'foo'}$\;and $\mathtt{'bar'}$\;wouldn't make sense i.e. there is no defined output for
these inputs. To prevent calls like these, some functional programming
languages introduced types. Types contain only certain values. For example, the
type for truth values contains only the values for true and false. In Haskell
we can define it like the following:
\begin{minted}[fontsize=\scriptsize]{haskell}
data Bool = True | False
\end{minted}
This says we can construct values of type $\mathtt{Bool}$\;with the constructors $\mathtt{True}$
and $\mathtt{False}$. These types defined with constructors are called inductive types. We can
then define $\mathtt{or}$\;like this:
\begin{minted}[fontsize=\scriptsize]{haskell}
or :: Bool -> Bool -> Bool
or True _    = True
or _    True = True
or _    _    = False
\end{minted}
Here, we just list equations that define what the output for a given input is.
For example, in the first equation, we say if the first value is constructed
with the constructor $\mathtt{True}$, we give back $\mathtt{True}$.  We don't care about the
second value, therefore we write $\mathtt{\_}$.  We are matching on the construction of
the input values.  Therefore, we call this method pattern matching.
If we call this function somewhere in the code on values that aren't of type
$\mathtt{Bool}$,  Haskell won't compile our code.  Instead, it gives back a type error.

If we now want to change $\mathtt{Bool}$\;to a three-valued logic, we have to add a
third constructor to $\mathtt{Bool}$. After that, we have to change every function which
pattern matches on $\mathtt{Bool}$. If there are a lot of those kinds of functions,
this would be a lot of repetitive work. If Haskell would have coinductive types,
this could be a lot less work. Coinductive types are types that are,
contrary to inductive types, defined over their destruction. So we could define
$\mathtt{Bool}$\;over its destructors. These would be or, and, etc.

Through this work, we will explain coinductive types using the examples of streams
and functions. Streams and functions will be generalized to partial streams
and the Pi type in dependently typed languages. Streams are lists that are
infinitely long. They are useful for modeling many IO interactions. For
example, a chat of a text messenger might be infinitely long. We can never
know if the chat is finished. This is of course limited by the hardware, but
we are interested in abstract models. Functions are used everywhere in
functional programming. In most of these languages, they are first-class
objects which are hardwired into the language. But in languages with
coinductive types, we can define them. If we only have inductive and
coinductive types, we get a symmetrical language. This is useful because then
we can change an inductive type to a coinductive one and vice versa. It is
straight forward to add functions which destruct an inductive type by
pattern matching on the constructor. But it is hard to add a new constructor.
Then, we add this constructor to every pattern matching on that type.
For coinductive types it's the other way around. For more on this, see
\cite{binder2019decomposition}. In the implemented syntax we can define streams
like the following:
\begin{lstlisting}
codata Stream$\langle$A : Set$\rangle$ : Set where
  Hd : Stream $\rat$ A
  Tl : Stream $\rat$ Stream
\end{lstlisting}
And functions like follows:
\begin{lstlisting}
codata Fun$\langle$A : Set, B : Set$\rangle$ : Set where
  Inst : (x : A) $\rat$ Fun $\rat$ B
\end{lstlisting}
We can generalize streams to partial streams as the following:
\begin{lstlisting}
codata PStr$\langle$A : Set$\rangle$ : (n : Conat) $\rat$ Set where
  Hd : (k : Conat) $\rat$ PStr (succ @ k) $\rat$ A
  Tl : (k : Conat) $\rat$ PStr (succ @ k) $\rat$ PStr @ k
\end{lstlisting}
These streams depend on co-natural numbers.  These are like natural numbers
with one additional element, infinity.  Therefore, partial streams have
their length encoded in their type. We can generalize functions to the Pi type
as follows:
\begin{lstlisting}
codata Pi$\langle$A : Set, B : (x : A) $\rat$ Set$\rangle$ : Set where
  Inst : (x : A) $\rat$ Pi $\rat$ B @ x
\end{lstlisting}
Here the result type can depend on the input value.

The rest of this thesis is structured as follows:

\begin{itemize}
\item Chapter \ref{sec:org5fdab5d} shows how coinductive types can be defined.
Here, we will define the stream and function type, as well as some
functions on the stream.

\item We will see in Chapter \ref{sec:orgbdcd039} how
coinductive types are defined in the dependently typed languages Coq and
Agda. We will see that we can define them as positive or negative
coinductive types. We will show why positive coinductive types lead to
problems.

\item In Chapter \ref{sec:org25524b6} we
see how they are defined by \cite{basold2016type}. With this theory we can
then define coinductive types which depend on values. But this theory does
not allow to define types that depend on types because the theory does not
include a type universe

\item We will then in Chapter \ref{sec:org312ed67} explain how this theory is
implemented. Implementing this type theory requires us to rewrite rules from
a declarative to an algorithmic form. It will also be possible to define
types depending on types.

\item At last, we implement the examples from \cite{basold2016type} in our syntax.
Here, we will see the reduction steps for recursion and corecursion. We will
conclude this section with the example of partial streams, which is a
coinductive type that depends on a value.
\end{itemize}

\chapter{Coinductive Types}
\label{sec:org5fdab5d}
Inductive types are defined via their constructors. Functions taking an
inductive type as an input can be defined via pattern-matching. Coinductive
types on the other hand are defined via their destructors. Functions that
have coinductive types as their output are implemented via copattern matching,
which was introduced in the paper \cite{abel2013copatterns}. In that paper
streams are defined like the following:

\begin{lstlisting}
record Stream A = { head : A,
                    tail : Stream A }
\end{lstlisting}

The $\mathtt{A}$\;in the definition should be a concrete
type\footnote{The type system in the paper doesn't have dependent types.}.
What differentiates this from regular record types (for example in Haskell) is
the recursive field tail. So they call it a recursive record. In a strict
language without coinductive types we could never instantiate such a type
because to do this we already need something of type $\mathtt{Stream\;A}$\;to fill in the
field $\mathtt{tail}$. The paper defines copattern matching to remedy this. With the
help of copattern matching, we can define functions that output expressions of
type $\mathtt{Stream\;A}$. As an example, we look at the definition of repeat. This
function takes in a value of type $\mathtt{Nat}$\;and generates a stream that just
infinitely repeats it.

\begin{lstlisting}
repeat : Nat $\rat$ Stream Nat
head (repeat x) = x
tail (repeat x) = repeat x
\end{lstlisting}

As we can see, copattern matching works via observations i.e. we define what
should be the output of the fields applied to the result of the function.
Because inhabitants of $\mathtt{Stream}$\;are infinitely long we can't print out a
stream. Because of this we also consider each expression which has a
coinductive type as a value. To get a subpart of this value we use observers.
For example, we can look at the third value of $\mathtt{repeat\;2}$\;via $\mathtt{head\;(tail\;(tail\;(repeat\;2)))}$
which should evaluate to 2. We can also implement a
function that looks at the nth. value. Here it is:

\begin{lstlisting}
nth : Nat $\rat$ Stream A $\rat$ A
nth 0     x = head x
nth (S n) x = nth n (tail x)
\end{lstlisting}

In the implementation of $\mathtt{nth}$, we use ordinary pattern matching on the left-hand side and
destructors on the right-hand side. $\mathtt{nth\;3\;(repeat\;2)}$\;will output $\mathtt{2}$\;as expected.
Functions can also be defined via a recursive record.  It is defined as the
following:

\begin{lstlisting}
record A $\rat$ B = { apply : A $\leadsto$ B }
\end{lstlisting}

Here, we differentiate between our defined function $\mathtt{A\;\rat\;B}$\;and $\mathtt{\leadsto}$\;in the
destructor. Constructor applications or, as is the case here, destructor
applications are not the same as function applications. In the
paper $\mathtt{f\;x}$\;means $\mathtt{apply\;f\;x}$. We will also use this convention in the
following. In fact, we already used it in the definitions of the functions
$\mathtt{repeat}$\;and $\mathtt{nth}$. $\mathtt{nth\;0\;x\;=\;head\;x}$\;is just a nested copattern. We can also write it
with $\mathtt{apply}$\;like so: $\mathtt{apply\;(apply\;nth\;0)\;x\;=\;head\;x}$. Here, we use currying.
So the first apply is the sole observer of type $\mathtt{Stream\;A\;\rat\;A}$\;and the second
of type $\mathtt{Nat\;\rat\;(Stream\;A\;\rat\;A)}$.

\chapter{Coinductive Types in Dependently Typed Languages}
\label{sec:orgbdcd039}
In this section, we will look at how coinductive types are implemented in
dependently typed languages. In dependently typed languages types can depend on
values. The classical example of such a type is the type for vectors. Vectors are like
lists, except their length is contained in their type. For example, a vector of
natural numbers of length 2 has type $\mathtt{Vec\;Nat\;2}$. This type depends on two
things. Namely the type $\mathtt{Nat}$\;and the value $\mathtt{2}$, which is itself of type $\mathtt{Nat}$.
We can define vectors in Coq as follows:
\begin{minted}[fontsize=\scriptsize]{coq}
Inductive Vec (A : Set) : nat -> Set :=
  | Nil : Vec A 0
  | Cons : forall {k : nat}, A -> Vec A k -> Vec A (S k).
\end{minted}
Contrary to a list the type constructor $\mathtt{Vec}$\;has a second argument $\mathtt{nat}$.
This is the already mentioned length of the vector. A Vector has two
constructors. One for an empty vector called $\mathtt{Nil}$\;and one to append an
element at the front of a vector called $\mathtt{Cons}$. $\mathtt{Nil}$\;just returns a vector
of length $\mathtt{0}$. And $\mathtt{Cons}$\;gets an $\mathtt{A}$\;and a vector of length $\mathtt{k}$. It returns a
vector of length $\mathtt{S\;k}$\;($\mathtt{S}$\;is just the successor of k). This type can also be
defined in Agda as follows:
\begin{minted}[fontsize=\scriptsize]{agda}
data Vec (A : Set) : ℕ → Set where
  Nil : Vec A 0
  Cons : {k : ℕ} → A → Vec A k → Vec A (suc k)
\end{minted}
One advantage of vectors compared to lists is that we can define a total function
(a function which is defined for every input) that takes the head of a
vector. This function can't be total for lists, because we cannot know if the
input list is empty. An empty list has no head. For vectors, we can enforce this
in Coq like follows:
\begin{minted}[fontsize=\scriptsize]{coq}
Definition hd {A : Set} {k : nat} (v : Vec A (S k)) : A :=
  match v with
  | Cons _ x _ => x
  end.
\end{minted}
We just pattern match on $\mathtt{v}$.  The only pattern is for the $\mathtt{Cons}$\;constructor.  The $\mathtt{Nil}$\;constructor
is a vector of length 0.  But $\mathtt{v}$\;has type $\mathtt{Vec\;A\;(S\;k)}$.  So it can't be a vector of length 0.
In Agda the function looks like follows:
\begin{minted}[fontsize=\scriptsize]{agda}
hd : {A : Set} {k : ℕ} → Vec A (suc k) → A
hd (cons x _) = x
\end{minted}
That types can depend on terms makes it necessary to ensure that functions
\linebreak terminate. Otherwise, typechecking wouldn't be guaranted to terminate. If we have a
function \linebreak $\mathtt{f\;:\;Nat\;\rat\;Nat}$\;and we want to check a value $\mathtt{a}$\;against a type
$\mathtt{Vec\;(f\;1)}$\;we have to know what $\mathtt{f\;1}$\;evaluates to. So $\mathtt{f}$\;has to terminate.
We check termination in Coq via a structurally decreasing argument. An argument
is structurally decreasing if it is structurally smaller in a recursive call.
Structurally smaller means it is a recursive occurrence in a constructor. As an
example, we look at addition of natural numbers. Natural numbers are defined
in Coq like follows:
\begin{minted}[fontsize=\scriptsize]{coq}
Inductive nat : Set :=
| O : nat
| S : nat -> nat.
\end{minted}
$\mathtt{O}$\;is the constructor for 0 and $\mathtt{S}$\;is the successor of its argument. Here,
the recursive argument to $\mathtt{S}$\;is structurally smaller than $\mathtt{S}$\;applied to it i.e.
$\mathtt{n}$\;is structurally smaller than $\mathtt{S\;n}$. Then, we can define addition like follows:
\begin{minted}[fontsize=\scriptsize]{coq}
Fixpoint add (n m : nat) : nat :=
match n with
| O => m
| S p => S (add p m)
end.
\end{minted}
In the recursive call, the first argument is structurally decreasing. The
expression $\mathtt{p}$\;is smaller than the expression $\mathtt{s\;p}$. So Coq accepts this
definition. The classical example of a function where an argument is
decreasing but not structurally decreasing is Quicksort. A naive implementation
of Quicksort in Coq would be the following:
\begin{minted}[fontsize=\scriptsize]{coq}
Fixpoint quicksort (l : list nat) : list nat :=
match l with
| nil => nil
| cons x xs => match split x xs with
              | (lower, upper) => app (quicksort lower) (cons x (quicksort upper))
              end
end.
\end{minted}
Here, $\mathtt{split}$\;is just a function that gets a number and a list of numbers. It
gives back a pair of two lists where the elements of the left list are all
elements of the input list which are smaller than the input number and the
right these which are bigger. It is clear that these lists can't be longer
than the input list. So $\mathtt{lower}$\;and $\mathtt{upper}$\;can't be longer than $\mathtt{xs}$. Here
$\mathtt{xs}$\;is structurally smaller than the input $\mathtt{cons\;x\;xs}$. So $\mathtt{lower}$\;and $\mathtt{upper}$
are smaller than the input. Therefore, we know that $\mathtt{quicksort}$\;is terminating.
But Coq won't accept this definition, because no argument is structurally decreasing.

For coinductive types termination means that functions that produce them
should be productive. Productive functions produce in each step a
new part of the infinitely large coinductive type.

In Section \ref{sec:orgb36caa9} we will look at the implementation of
coinductive types in Coq. There are two ways to define coinductive types in
Coq. The older way uses positive coinductive types. This is known to violate
subject reduction. Therefore, it is highly discouraged to use them. To fix
this the new way uses negative coinductive types. In Section \ref{sec:org174b8e2}
we look at the implementation of coinductive types in Agda. Agda also has these two ways of
defining such types. One special thing about it, is that Agda implements
copattern matching. To help Agda with termination checking we can use sized
types. We will explain them in Section \ref{sec:orgb49e466}.
\section{Coinductive Types in Coq}
\label{sec:orgb36caa9}
There are two approaches to define coinductive types in Coq. The older one,
positve coinductive types which are defined via constructors, is
described in section \ref{sec:orgd7a50c1}. The newer and recommended one is
described in Section \ref{sec:orgf6a06b9}. They are defined using
primitive records (a relatively new feature of Coq). Therefore, they are
called negative coinductive Types.

\subsection{Positive Coinductive Types}
\label{sec:orgd7a50c1}
Positive coinductive types are defined over constructors in Coq.  The keyword
\linebreak $\mathtt{CoInductive}$\;is used to mark the definition as a coinductive type.
This is the only syntactical difference from the definition of inductive
types. For example, streams are defined like the following:

\begin{minted}[fontsize=\scriptsize]{coq}
CoInductive Stream (A : Set) : Set :=
  Cons : A -> Stream A -> Stream A.
\end{minted}

If this were an inductive type we couldn't generate a value of this type. To
generate values of coinductive types Coq uses guarded recursion. Guarded
recursion checks if the recursive call to the function occurs as an argument
to a coinductive constructor. In addition to the guard condition, the
constructor can only be nested in other constructors, fun or match
expressions. With all of this in mind we can define $\mathtt{repeat}$\;like the
following:

\begin{minted}[fontsize=\scriptsize]{coq}
CoFixpoint repeat (A : Set) (x : A) : Stream A := Cons A x (repeat A x).
\end{minted}

Then, we can produce the constant zero stream with $\mathtt{repeat\;nat\;0}$. If we used
$\mathtt{Fixpoint}$\;instead of $\mathtt{CoFixpoint}$\;Coq wouldn't accept our code. It rejects
it because there is no argument which is structural decreasing. $\mathtt{x}$\;stays
always the same. Functions defined with $\mathtt{CoFixpoint}$\;on the other hand only
check the previously mentioned conditions. It sees that the recursive call $\mathtt{repeat\;A\;x}$
occurs as an argument to the constructor $\mathtt{Cons}$\;of the coinductive type
$\mathtt{Stream}$. This constructor is also not nested. So our definition is accepted.

We can use the normal pattern matching of Coq to destruct a coinductive type.
We define $\mathtt{nth}$\;like the following:

\begin{minted}[fontsize=\scriptsize]{coq}
Fixpoint nth (A : Set) (n : nat) (s : Stream A) {struct n} : A :=
  match s with
    Cons _ a s' =>
    match n with 0 => a | S p => nth A p s' end
  end.
\end{minted}

The guard condition is necessary to ensure every expression is terminating.
If we didn't have the guard condition we could define the following:

\begin{minted}[fontsize=\scriptsize]{coq}
CoFixpoint loop (A : Set) : Stream A = loop A.
\end{minted}

Here, the recursive call doesn't occur in a constructor. So the guard
condition is violated. With this definition the expression $\mathtt{nth\;0\;loop}$
wouldn't terminate. The function $\mathtt{nth}$\;would try to pattern match on $\mathtt{loop}$. But to
succeed in that $\mathtt{loop}$\;has to unfold to something of the form $\mathtt{Cons\;a\;?}$
which it never does. So $\mathtt{nth\;0\;loop}$\;will never evaluate to a value.

We illustrate the purpose of the other conditions on an example taken from
\cite{chlipala2013certified}.  First, we implement the function $\mathtt{tl}$\;like so:

\begin{minted}[fontsize=\scriptsize]{coq}
Definition tl A (s : Stream A) : Stream A :=
  match s with
  | Cons _ _ s' => s'
  end.
\end{minted}

This is just one normal pattern match on $\mathtt{Stream}$.  If we didn't have the
other condition we could define the following:

\begin{minted}[fontsize=\scriptsize]{coq}
CoFixpoint bad : Stream nat := tl nat (Cons nat 0 bad).
\end{minted}

This doesn't violate the guard condition.  The recursive call $\mathtt{bad}$\;is an
argument to the constructor $\mathtt{Cons}$.  But the constructor is nested in a
function.  If we would allow this, $\mathtt{nth\;0\;bad}$\;would loop forever.  To
understand why we first unfold $\mathtt{tl}$\;in $\mathtt{bad}$.  So we get:

\begin{minted}[fontsize=\scriptsize]{coq}
nth 0 (cofix bad : Stream nat :=
         match (Cons 0 bad) with
         | Cons _ s' => s'
         end)
\end{minted}

We can now simplify this to just:

\begin{minted}[fontsize=\scriptsize]{coq}
nth 0 (cofix bad : Stream nat := bad)
\end{minted}

After that $\mathtt{bad}$\;isn't any more an argument to a constructor.  Here, we can also
see easily that the expression $\mathtt{cofix\;bad\;:\;Stream\;nat\;:=\;bad}$\;loops forever.
So we never get the value at position $\mathtt{0}$.

An important property of typed languages is subject reduction. Subject
reduction says if we evaluate an expression \(e_1\) of type \(t\) to an expression
\(e_2\), \(e_2\) should also be of type \(t\). With positive coinductive types subject
reduction no longer holds. We illustrate this by Oury's counterexample
\cite{oury2008}. First, we define the codata type $\mathtt{U}$\;as follows:

\begin{minted}[fontsize=\scriptsize]{coq}
CoInductive U : Set := In : U -> U.
\end{minted}

We can now define a value of $\mathtt{U}$\;with the following $\mathtt{CoFixpoint}$\;like so:

\begin{minted}[fontsize=\scriptsize]{coq}
CoFixpoint u : U := In u.
\end{minted}

This generates an infinite succession of $\mathtt{In}$.  We use the function $\mathtt{force}$
to force $\mathtt{u}$\;to evaluate one step i.e. $\mathtt{u}$\;becomes $\mathtt{In\;u}$.

\begin{minted}[fontsize=\scriptsize]{coq}
Definition force (x: U) : U :=
  match x with
    In y => In y
  end.
\end{minted}

The same trick will be used to define $\mathtt{eq}$\;which states that $\mathtt{x}$\;is
propositionally equal to $\mathtt{force\;x}$.

\begin{minted}[fontsize=\scriptsize]{coq}
Definition eq (x : U) : x = force x :=
  match x with
    In y => eq_refl
  end.
\end{minted}

The function $\mathtt{eq}$\;matches on $\mathtt{x}$, reducing to $\mathtt{In\;y}$. Then, the new goal
becomes $\mathtt{In\;y\;=\;force\;(In\;y)}$. The term $\mathtt{force\;(In\;y)}$\;evaluates to $\mathtt{In\;y}$, as $\mathtt{force}$
just pattern matches on $\mathtt{In\;y}$. So the final goal is $\mathtt{In\;y\;=\;In\;y}$\;which
can be shown by $\mathtt{eq\_refl}$. The expression $\mathtt{eq\_refl}$\;is a constructor for $\mathtt{=}$
where both sides of $\mathtt{=}$\;are exactly the same. If we now instantiate $\mathtt{eq}$\;with
$\mathtt{u}$\;we become $\mathtt{eq\;u}$.

\begin{minted}[fontsize=\scriptsize]{coq}
Definition eq_u : u = In u := eq u
\end{minted}

But $\mathtt{u}$\;is not definitional equal to $\mathtt{In\;u}$.  As mentioned above expressions
with a coinductive type are always values to prevent infinite evaluation.
Both $\mathtt{In\;u}$\;and $\mathtt{u}$\;are values. But values are only
definitional equal if they are exactly the same.  The next section will
solve this problem through negative coinductive types.

\subsection{Negative Coinductive Types}
\label{sec:orgf6a06b9}
In Coq 8.5. primitive records were introduced. With this, it is now possible
to define types over their destructors. So we can have negative, especially
negative coinductive, types in Coq. With primitive records we can define
streams like the following:

\begin{minted}[fontsize=\scriptsize]{coq}
CoInductive Stream (A : Set) : Set :=
  Seq { hd : A; tl : Stream A }.
\end{minted}

Now we can define $\mathtt{repeat}$\;over the fields of $\mathtt{Stream}$.

\begin{minted}[fontsize=\scriptsize]{coq}
CoFixpoint repeat (A : Set) (x : A) : Stream A :=
  {| hd := x; tl := repeat A x|}.
\end{minted}

To define $\mathtt{repeat}$\;we must define what is the head of the constructed stream
and its tail.  The guard condition now says that corecursive
occurrences must be guarded by a record field.  We can see that the
corecursive call $\mathtt{repeat}$\;is a direct argument to the field $\mathtt{tl}$\;of the
corecursive type $\mathtt{Stream\;A}$.  This means that Coq accepts the above definition.
If we want to access parts of a stream we use the destructors $\mathtt{hd}$\;and
$\mathtt{tl}$.  With them, we can define $\mathtt{nth}$\;again for the negative stream.

\begin{minted}[fontsize=\scriptsize]{coq}
Fixpoint nth (A : Set) (n : nat) (s : Stream A) : list A :=
  match n with
  | 0 => s.(hd A)
  | S n' => nth A n' s.(tl A)
  end.
\end{minted}

With negative coinductive types, we can't form the above-mentioned
counterexample to subject reduction anymore, because we can't pattern match
on negative types. Oury's example becomes.

\begin{minted}[fontsize=\scriptsize]{coq}
CoInductive U := { out : U }.
\end{minted}

$\mathtt{U}$\;is now defined via its destructor $\mathtt{out}$, instead of its constructor
 $\mathtt{in}$. Then, $\mathtt{in}$\;\linebreak becomes just a function. In fact, it's just a
 definition because we don't recurse or corecurse on the argument $\mathtt{y}$.

\begin{minted}[fontsize=\scriptsize]{coq}
Definition in (y : U) : U := {| out := y |}.
\end{minted}

We define it over the only field $\mathtt{out}$.  When we put a $\mathtt{y}$\;in then we get
the same $\mathtt{y}$\;out.  We can also again define $\mathtt{u}$.

\begin{minted}[fontsize=\scriptsize]{coq}
CoFixpoint u : U := {| out := u |}.
\end{minted}

With coinductive types, it is now possible to define the pi type (the dependent
function type).

\begin{minted}[fontsize=\scriptsize]{coq}
CoInductive Pi (A : Set) (B : A -> Set) := { Apply (x : A) : B x }.
\end{minted}

The pi type is defined over its destructor $\mathtt{Apply}$.  If we evaluate $\mathtt{Apply}$
on a value of $\mathtt{Pi}$\;(which is a function) and an argument, we get the result
i.e. we apply the value to the function.  It looks like the pi type becomes definable
in Coq.  But we are cheating.  The type of $\mathtt{Apply}$\;is already a pi type because
we identify constructors and destructors with functions. We will see that
the theory \cite{basold2016type} avoids this identification. To define a
function we use $\mathtt{CoFixpoint}$. As a simple nonrecursive, nondependent example
we use the function $\mathtt{plus2}$.

\begin{minted}[fontsize=\scriptsize]{coq}
CoFixpoint plus2 : Pi nat (fun _ => nat) :=
  {| Apply x  := S (S x) |}.
\end{minted}

If we apply (i.e. call the destructor $\mathtt{Apply}$) a $\mathtt{x}$\;to $\mathtt{plus2}$\;it gives back
$\mathtt{S\;(S\;x)}$.  Which is twice the successor on $\mathtt{x}$.  So we add 2 to $\mathtt{x}$.  We
use $\mathtt{\_}$\;here because $\mathtt{plus2}$\;is not a dependent function i.e. the result
type $\mathtt{nat}$\;doesn't depend on the input value.  To define functions with more
than one argument we just use currying i.e. we use the type $\mathtt{Pi}$\;as the
second argument to $\mathtt{Pi}$. For example, a 2-ary non-dependent function from $\mathtt{A}$
and $\mathtt{B}$\;to $\mathtt{C}$\;would have type $\mathtt{Pi\;A\;(fun\;\_\;=>\;Pi\;B\;(fun\;\_\;=>\;C))}$.  It
would be fortunate if we could define $\mathtt{plus}$\;like the following:

\begin{minted}[fontsize=\scriptsize]{coq}
CoFixpoint plus : Pi nat (fun _ => Pi nat (fun _ => nat)) :=
  {| Apply := fun (n : nat)  =>
       match n with
       | O => {| Apply (m : nat) := m |}
       | S n' => {| Apply m := S (Apply _ _ (Apply _ _  plus n') m) |}
       end
  |}.
\end{minted}

But Coq doesn't accept this definition since it is violates the guard condition. The
expression $\mathtt{plus\;n'}$\;is not a direct argument of the field $\mathtt{Apply}$. The
definition should terminate because we are decreasing $\mathtt{n}$\;and the case for
$\mathtt{0}$\;is accepted. In the case of $\mathtt{0}$, there is no recursive call.

We can also define a dependent function.  We define append2Units like
follows
\begin{minted}[fontsize=\scriptsize]{coq}
CoFixpoint append2Units : Pi nat
                             (fun n => Pi (Vec unit n)
                                       (fun _ => Vec unit (S (S n)))) :=
  {| Apply n := {| Apply v := Cons _ tt (Cons _ tt v) |} |}.
\end{minted}
This just appends 2 units at a vector of length $\mathtt{n}$.  Here, the second
argument and the result depend on the first argument i.e. the first argument
is the length of the input vector and the output vector is this length plus
two.

\section{Coinductive Types in Agda}
\label{sec:org174b8e2}
In Agda coinductive types were first also introduced as positive types. In
Section \ref{sec:orga9a7203} we will look at them in
detail. In Section \ref{sec:org95ffc71} we describe the correct
way to implement coinductive types in Agda. There are functions which
terminate but are rejected by the typechecker. To allow more functions we
can use a unique feature of Agda, sized types. They are described in Section
\ref{sec:orgb49e466}.

\subsection{Positive Coinductive Types in Agda}
\label{sec:orga9a7203}
Agda doesn't have a special keyword to define coinductive types like Coq.  It
uses the type constructor $\mathtt{\infty}$\;to mark arguments to constructors as coinductive.
This type constructor says that the computation of arguments of this type is suspended.
So Agda ensures productivity over typechecking. We define streams like so.

\begin{minted}[fontsize=\scriptsize]{agda}
data Stream (A : Set) : Set where
  cons : A → ∞ (Stream A) → Stream A
\end{minted}

Here, the tail of the stream. is marked with $\mathtt{\infty}$. Because the tail is
infinitely long (we don't have a constructor of an empty stream) we can't
compute it completely, so we suspend the computation. We can delay a
computation with the constructor $\mathtt{\sharp}$\;and force it with the function
$\mathtt{\flat}$. Their types are given below.

\begin{minted}[fontsize=\scriptsize]{agda}
♯_ : ∀ {a} {A : Set a} → A → ∞ A
♭  : ∀ {a} {A : Set a} → ∞ A → A
\end{minted}

We can now again define our usual functions.  We begin with $\mathtt{repeat}$.

\begin{minted}[fontsize=\scriptsize]{agda}
repeat : {A : Set} → A → Stream A
repeat x = cons x (♯ (repeat x))
\end{minted}

We first apply $\mathtt{cons}$\;to $\mathtt{x}$. So the head of the stream is $\mathtt{x}$. We then apply
it to the corecursive call $\mathtt{repeat}$. So the tail will be a repetition of
$\mathtt{xs}$. We have to call the $\mathtt{repeat}$\;with $\mathtt{\sharp}$\;to suspend the computation.
Otherwise, the code doesn't typecheck. If we would write this function
without $\mathtt{\sharp}$\;on a stream which has no $\mathtt{\infty}$\;on the second argument of
$\mathtt{cons}$, the function would run forever. In fact, the termination checker won't
allow us to write such a function. We can also write $\mathtt{nth}$\;again, which
consumes a stream.

\begin{minted}[fontsize=\scriptsize]{agda}
nth : {A : Set} → ℕ → Stream A → A
nth 0       (cons x _)  = x
nth (suc n) (cons _ xs) = nth n (♭ xs)
\end{minted}

Here, we have to use $\mathtt{\flat}$\;on the right-hand side of the second case, to
force the computation of the tail of the input stream.  We have to do that
because $\mathtt{nth}$\;wants a stream, not a suspended stream.
Productivity on coinductive types like $\mathtt{Stream}$\;is checked by only allowing non
decreasing recursive calls behind the $\mathtt{\sharp}$\;constructor.

\subsection{Negative Coinductive Types in Agda}
\label{sec:org95ffc71}
In Agda we can also define negative coinductive types.  This is the
recommended way.  Agda implements the previously mentioned copattern matching.
We can define a record with the keyword $\mathtt{record}$.  We use the keyword $\mathtt{coinductive}$
to make it possible to define recursive fields.  Stream is defined as the
following:

\begin{minted}[fontsize=\scriptsize]{agda}
record Stream (A : Set) : Set where
  coinductive
  field
    hd : A
    tl : Stream A
\end{minted}

A Stream has 2 fields. The field $\mathtt{hd}$\;is the head of the stream. It has type
$\mathtt{A}$. The field $\mathtt{tl}$\;is the tail of the stream. It is another stream, so it
has type $\mathtt{Stream\;A}$. $\mathtt{tl}$\;is a recursive field. So Agda wouldn't accept the
definition without $\mathtt{coinductive}$. A stream can never be empty. So every
stream has to have a head (a field $\mathtt{hd}$). So the tail of a stream can never
be empty. Therefore, every stream is infinitely long. We can now define
$\mathtt{repeat}$\;with copattern matching.

\begin{minted}[fontsize=\scriptsize]{agda}
repeat : ∀ {A : Set} → A → Stream A
hd (repeat x) = x
tl (repeat x) = repeat x
\end{minted}

We have to copattern match on every field of $\mathtt{Stream}$, namely $\mathtt{hd}$\;and $\mathtt{tl}$.
Because Agda is total it won't accept non-exhaustive (co)pattern matches
like Haskell. In the first copattern we define what the head of $\mathtt{repeat\;x}$
is. It is $\mathtt{x}$\;because we repeat $\mathtt{x}$\;infinitely often. In the second
copattern we define what the tail of the stream is. The tail is just \texttt{repeat
\;\;\;\;x}. We can use normal pattern matching and the destructors for functions
that consume streams. We define $\mathtt{nth}$\;like the following:

\begin{minted}[fontsize=\scriptsize]{agda}
nth : ∀ {A : Set} → ℕ → Stream A → A
nth zero s = hd s
nth (suc n) s = nth n (tl s)
\end{minted}

Here, we just pattern match on the first argument (excluding the implicit
argument of the type).  If it is zero the result is just the head of the
stream.  If it is \(n+1\) the result is the recursive call of $\mathtt{nth}$\;on $\mathtt{n}$\;and
$\mathtt{tl\;s}$.  Agda accepts this code because it is structural decreasing on the
first (or second if we count the implicit) argument.

We can also define the $\mathtt{Pi}$\;type.  We use $\mathtt{\_\$\_}$\;as the apply operator.

\begin{minted}[fontsize=\scriptsize]{agda}
record Pi (A : Set) (B : A → Set) : Set where
  field _$_ : (x : A) → B x
  infixl 20 _$_
open Pi
\end{minted}

Like in Coq we are using the first-class pi type to define the pi type. Agda
doesn't define the first-class pi type lkie that.  We
can also define a function $\mathtt{plus2}$\;in Agda.

\begin{minted}[fontsize=\scriptsize]{agda}
plus2 : ℕ →' ℕ
plus2 $ x = suc (suc x)
\end{minted}

We just use copattern matching to define it. If we apply a $\mathtt{x}$\;to $\mathtt{plus2}$\;we
get \linebreak $\mathtt{suc\;(suc\;x)}$. The type $\mathtt{\rightarrow'}$\;is the non-dependent
function which is is defined
using our pi type. Here it is:

\begin{minted}[fontsize=\scriptsize]{agda}
_→'_ : Set → Set → Set
A →' B = Pi A (λ _ → B)
infixr 20 _→'_
\end{minted}

In Agda it becomes possible to define plus. We just use nested copattern
matching.

\begin{minted}[fontsize=\scriptsize]{agda}
plus : ℕ →' ℕ →' ℕ
plus $ 0       $ m = m
plus $ (suc n) $ m = suc (plus $ n $ m)
\end{minted}

If we change $\mathtt{\rightarrow'}$\;to $\mathtt{\rightarrow}$\;and remove $\mathtt{\$}$\;we get the
standard definition for plus in Agda. We can also define a dependent function
$\mathtt{repeatUnit}$\;like follow:
\begin{minted}[fontsize=\scriptsize]{agda}
repeatUnit : Pi ℕ (λ n → Vec ⊤ n)
repeatUnit $ 0     = nil
repeatUnit $ suc n = tt :: (repeatUnit $ n)
\end{minted}
This function gives back a vector with the length of the input, where every element
is unit.

\subsection{Termination Checking with Sized Types}
\label{sec:orgb49e466}
They are many functions which are total but are not accepted by Agda's
termination checker. In fact, in any total language, there have to be such
functions. We can show that by trying to list all total functions. The
following table lists functions per row. The columns say what the output of
the functions for the given input is.
\begin{center}
\begin{tabular}{lrrrrl}
 & 1 & 2 & 3 & 4 & \(\dots\)\\
\hline
\(f_1\) & 2 & 7 & 8 & 6 & \(\dots\)\\
\(f_2\) & 4 & 4 & 6 & 19 & \(\dots\)\\
\(f_3\) & 6 & 257 & 1 & 2 & \(\dots\)\\
\(f_4\) & 7 & 121 & 23188 & 2313 & \(\dots\)\\
\(\vdots\) & \(\vdots\) & \(\vdots\) & \(\vdots\) & \(\vdots\) & \(\ddots\)\\
\end{tabular}
\end{center}
We can now define a function \(g(n)=f_n(n)+1\) this function is total and not
in the list because it is different from any function in the list for at
least one input. As an example of such a function, we could try to define
division with rest on natural numbers like the following:

\begin{minted}[fontsize=\scriptsize]{agda}
_/_ :  ℕ → ℕ → ℕ
zero / y = zero
suc x / y = suc ( (x - y) / y)
\end{minted}

The problem with this definition is that Agda doesn't know that $\mathtt{x-y}$\;is
smaller than $\mathtt{x+1}$, which is clearly the case ($\mathtt{x}$\;and $\mathtt{y}$\;are positive).  This
definition would work perfectly fine in a language without termination
checking (like Haskell).  Agda only checks if an argument is structurally
decreasing.  Here, it is neither the case for $\mathtt{x}$\;nor for $\mathtt{y}$.

To remedy this problem sized types were introduced first to Mini-Agda (a
language specifically developed to explore them) by \cite{abel2010miniagda}.
Later, they got introduced to Agda itself. Sized types allow us to annotate data
with their size. Functions can use these sizes to check termination and
productivity.

We can now define the natural numbers depending on a size argument.
\begin{minted}[fontsize=\scriptsize]{agda}
data ℕ (i : Size) : Set where
  zero : ℕ i
  suc : ∀{j : Size< i} → ℕ j → ℕ i
\end{minted}
The natural number now depends on the size $\mathtt{i}$.  The constructor $\mathtt{zero}$\;is of
arbitrary size $\mathtt{i}$.  The constructor $\mathtt{suc}$\;gets a size $\mathtt{j}$\;which is smaller than $\mathtt{i}$, a
natural number of size $\mathtt{j}$\;and gives back a natural number of size $\mathtt{i}$.  This
means the size of the input is smaller than the size of the output.  For
inductive types, size is an upper bound on the number of constructors.  With
$\mathtt{suc}$\;we add a constructor so the size has to increase  $\mathtt{i}$.  We can now define
subtraction on these sized natural numbers.
\begin{minted}[fontsize=\scriptsize]{agda}
_-_ : {i : Size} → ℕ i → ℕ ∞ → ℕ i
zero    - _      = zero
n       - zero   = n
(suc n) - (suc m) = n - m
\end{minted}
Through the sized annotations we know now that the result isn't larger than
the first input.  \(\infty\) means that the size isn't bound.  If the first
argument is zero the result is also zero, which has the same type.  If the
second argument is zero we return just the first.  In the last, case both
arguments are non-zero.  We call subtraction recursively on the predecessors
of the inputs.  Here, the size and both arguments are smaller.  So the
function terminates.  Though the type is smaller than \(i\), the result type
checks because sizes are upper bounds.  We can now define division.
\begin{minted}[fontsize=\scriptsize]{agda}
_/_ : {i : Size} → ℕ i → ℕ ∞ → ℕ i
zero  / _ = zero
suc x / y = suc ( (x - y) / y)
\end{minted}
From the definition of $\mathtt{suc}$\;we know that the size of $\mathtt{x}$\;is smaller than $\mathtt{i}$.
Because the result of $\mathtt{-}$\;has the same size as its first input (here $\mathtt{x}$),
we also know that $\mathtt{(x\;-\;y)}$\;has the same size as $\mathtt{x}$. Therefore, $\mathtt{(x\;-\;y)}$\;is
smaller than $\mathtt{suc\;x}$\;and the function is decreasing on the first argument.
Also, Agda accepts this definition.

We can also use sized types for coinductive types.  To show this we will
define the hamming function.  This produces a stream of all composites of two
and three in order.  First, we will define the sized stream type.
\begin{minted}[fontsize=\scriptsize]{agda}
record Stream (i : Size) (A : Set) : Set where
  coinductive
  field
    hd : A
    tl : ∀ {j : Size< i} → Stream j A
open Stream
\end{minted}
This stream has a new parameter of type $\mathtt{Size}$. This size gives the minimal
definition depth of the stream. The definition depth says how often we can
destruct the stream without diverging. If we take the tail of a stream, the
output stream's depth would be one smaller. Because in Agda coinductive types
can't have indexes, we can only say that its depth is smaller. We will now
define some helper functions for the hamming function. First, we need a cons
function.
\begin{minted}[fontsize=\scriptsize]{agda}
cons : {i : Size} {A : Set} → A -> Stream i A → Stream i A
hd (cons x _)  = x
tl (cons _ xs) = xs
\end{minted}
This just appends an element at the front of the stream.  Because the output
stream's depth is larger than the input and the size is a minimum, we can give
the output the same size parameter as the input.  Now we will define map over
streams.
\begin{minted}[fontsize=\scriptsize]{agda}
map : {A B : Set} {i : Size} → (A → B) → Stream i A → Stream i B
hd (map f xs) = f (hd xs)
tl (map f xs) = map f (tl xs)
\end{minted}
This function just changes the content of the stream so the size stays the
same.  The last helper function we need is the merge function.
\begin{minted}[fontsize=\scriptsize]{agda}
merge : {i : Size} → Stream i ℕ → Stream i ℕ → Stream i ℕ
hd (merge xs ys) = hd xs ⊓ hd ys
tl (merge xs ys) = if ⌊ hd xs ≤? hd ys ⌋
                   then cons (hd ys) (merge (tl xs) (tl ys))
                   else cons (hd xs) (merge (tl xs) (tl ys))
\end{minted}
This function just merges two streams. It always compares one element of
each stream with each other and puts the bigger after the smaller. This is
clear in the case for $\mathtt{hd}$\;($\mathtt{\sqcup}$\;is just the binary minimum function in
Agda). In the $\mathtt{tl}$\;case we just compare the heads of the stream and construct
the tail with $\mathtt{cons}$\;accordingly. Both input streams have a minimal
definition depth of $\mathtt{i}$. Because $\mathtt{cons}$\;isn't destructing the stream (the
minimal depth doesn't get smaller) we can say that the minimum depth of the
output also won't get smaller. With all this function we can now define the
ham function. Here it is:
\begin{minted}[fontsize=\scriptsize]{agda}
ham : {i : Size} → Stream i ℕ
hd ham = 1
tl ham = (merge (map (_*_ 2) ham) (map (_*_ 3) ham))
\end{minted}
None of the used function is destructing the stream, so this definition gets
accepted.

With sized types, we can define many total algorithm, which don't have a
structurally decreasing argument, in a total language. In contrary to the
Bove Capretta method \cite{bove2005modelling}, we don't have to change the structure of the
algorithm.

\chapter{Type Theory based on Dependent Inductive and Coinductive Types}
\label{sec:org25524b6}
In the paper \cite{basold2016type} Basold and Geuvers develop a type theory,
where inductive types and coinductive types can depend on values. User-defined
inductive and coinductive types are the only types in the system. For example,
we can, in contrast to the coinductive types of Coq and Agda, define streams
which depend on their definition length. The theory differentiates types from
terms. We don't have type universes, where every type can be seen as a term in
universe \(U_n\). Therefore, types can only depend on values, not on other
types. We only have builtin functions on the type level. These functions
abstract over terms. For example, \(\lambda x.A\) is a type where all
occurrences of the term variable \(x\) in \(A\) are bound. We will see that
functions on the term level are definable. We can apply types to terms. For
example, \(A @ t\) means we apply the type \(A\) to the term \(x\). Every type has a kind. A
kind is either \(*\) or \(\Gamma\rat*\). Here, \(\Gamma\) is a context which states
to what terms we can applythe type. For example, we can apply \(A\) of kind
\((x:B)\rat*\) only to a term of type \(B\). If we apply it to \(t\) of type \(B\), we
get a type of kind \(*\). We write \(\rat\) instead of \(\rightarrow\) to indicate,
that these are not functions. We can also apply a term to another term. For
example, \(t@s\) means we apply the term \(t\) to the term \(s\). Terms can also
depend on contexts. For example, if we have a term \(t\) of type \((x:A)\rat B\)
and apply it to a term \(s\) of type \(A\) we get a term of type \(B\).

We can also define our own types. \(\mu(X:\Gamma\rat*;\vv{\sigma};\vv{A})\) is
an inductive type and \(\nu(X:\Gamma\rat*;\vv{\sigma};\vv{A})\) is a coinductive
type. \(X\) is a variable that stands for the recursive occurrence of the type.
It has the same kind \(\Gamma\rat*\) as the defined type. The \(\vv{A}\) can
contain this variable. There are also contexts \(\vv{\Gamma}\), which are
implicit in the paper. \(\sigma_k\) and \(A_k\) can contain variables from
\(\Gamma_k\). \(\sigma_k\) is a context morphism from \(\Gamma_k\) to \(\Gamma\). A
context morphism is a sequence of terms, which depend on \(\Gamma_k\) and
instantiate \(\Gamma\). \(\vv{\sigma}\), \(\vv{A}\) and \(\vv{\Gamma}\) are of the
same length.

In this theory, we can define partial streams on some type \(A\) like the following:
\begin{align*}
&\text{PStr }A := \nu(X:(n:\text{Conat})\rat*;(\text{succ} @ n, \text{succ} @ n);(A, X @ n))\\
&\text{with } \Gamma_1 = (n:\text{Conat}) \text{ and } \Gamma_2 = (n:\text{Conat})
\end{align*}
Here, $\mathtt{succ}$\;is the successor on co-natural numbers.  Co-natural numbers are
natural numbers with one additional element, infinity. See \ref{sec:org2792671}
for their definition. Here, the first destructor is the head. It becomes a
stream with length \(\text{succ} @ N\) and returns an \(A\). The second destructor
is the tail. It becomes also a stream of length \(\text{succ} @ N\). It gives
back an \(X @ n\), which is a stream of length \(n\). We can also define the Pi
type from \(A\) to \(B\), where \(B\) can depend on \(A\).
\begin{align*}
&\Pi x:A.B := \nu(\_:*;\epsilon_1;B)\\
&\text{with } \Gamma_1 = (x:A)
\end{align*}
By \(\_\) we mean, we are ignoring this variable. \(\epsilon_1\) is one empty
context morphism.  So the only destructor gives back a \(B\) which can depend
on \(x\) of type \(A\).  It is the function application.

To construct an inductive type we use constructors (written
\(\alpha_k^{\mu(X:\Gamma\rat*;\vv{\sigma};\vv{A})}\) in the paper, which is the k-th
constructor of the given type).  We can destruct it with recursion (written
$\mathtt{rec}$\;\(\vv{(\Gamma_k.y_k).g_k}\)).  Coinductive types work the other way around.
We destruct them with destructors (written
\(\xi_k^{\nu(X:\Gamma\rat*;\vv{\sigma};\vv{A})}\)) and construct them with
corecursion (written $\mathtt{corec}$\;\(\vv{(\Gamma_k.y_k).g_k}\)).

We will give the rules for the theory in Section \ref{sec:org11f03b3} and a detailed
explanation of reduction in \ref{sec:orgedabc94}.

\chapter{Implementation}
\label{sec:org312ed67}
In this section, we look at the implementation details. We use the functional
programming language Haskell for implementing the theory. Haskell is a pure
language. This means functions which aren't in the IO monad have no side
effects. Because all essential parts of the implmentation are pure, we can
easily test it. Another feature of Haskell, which will get useful in our
implementation is pattern matching. We will see its usefulness in Section
\ref{sec:org11f03b3}.

In Section \ref{sec:org153f003} we will develop the abstract syntax of our language
from the raw syntax in the paper. Then, we rewrite the typing rules in \ref{sec:org11f03b3}. At last we look at the implementation of reduction in \ref{sec:orgedabc94}.
\section{Abstract Syntax}
\label{sec:org153f003}
In the following, we will describe the abstract syntax. In contrast to
\cite{basold2016type} we can't write anonymous inductive and coinductive types.
Instead, we define inductive and coinductive types via declarations, were we
give them names. In these declarations, we will give their
constructors/destructors together with their names. We can then refer to
the previously defined types. We will describe declarations in Section
\ref{sec:orga470b6c}. We will also be able to bind expressions to names. In Section
\ref{sec:org09156e6} we will define the syntax of expressions. This will mostly be in
one to one correspondence with the syntax of \cite{basold2016type}. Note
however, that we use the names of the constructors instead of anonymous
constructors together with their type and number. Also, the order of the
matches in $\mathtt{rec}$\;and $\mathtt{corec}$\;is irrelevant. We use the names of the
Con/Destructors to identify them. In the following Section \ref{sec:orgd840521}, we will
see how the examples from the paper look in our concrete syntax.
\subsection{Declarations}
\label{sec:orga470b6c}
The abstract syntax is given in Figure \ref{syntax-for-declarations}. With the
keywords data and codata we define inductive and coinductive types
respectively. After that, we will write the name. We can only use names that
aren't used already. Behind that, we can give a parameter context. This is a
type context. These types are not polymorphic. They are merely macros to make
the code more readable and allow the definition of nested types. If we want
to use these types we have to fully instantiate this context. These types can
occur everywhere in the definition where a type is expected. A (co)inductive
type can have a context which is written before an arrow. $\mathtt{Set}$\;stands for
type (or * in the paper). If a type doesn't have a context we omit the arrow.
We will also give names to every constructor and destructor. These names have
to be unique. Constructors and destructors also have contexts. Additionally,
they have one argument which can have a recursive occurrence of the type we
are defining. A constructor gives back a value of the type, where its
context is instantiated. This instantiation corresponds to the sigmas in the
paper. If we write a name before an equal sign we can bind the following
expression to the name. Every such defined name can depend on a parameter
context and an argument context. We write the parameter context like in the
case for data types behind the name. After that, we can give a term context
between round parenthesis.

\begin{figure}
 \begin{align*}
   \begin{array}{llll}
    N &:= &[A-Z][a-zA-Z0-9]* &\text{Names for types,}\\
         & &&          \text{constructors}\\
         & &&          \text{and destructors}\\
    n &:= &[a-z][a-zA-Z0-9]* &\text{Names for expressions}\\
    EV &:= &x,y,z,\dots &\text{Expression variables} \\
    TV &:= &X,Y,Z,\dots &\text{Type expression} \\
         & &&          \text{variables}\\
    PV  &:= &A,B,C,\dots &\text{Parameter variables} \\
    EC &:= &\lozenge &\text{Expression Context} \\
           &| &\text{(} EV \text { : } TV (,EV \text{ : } TV)*\text{)}& \\
    PC &:= &\langle\rangle &\text{Parameter Context} \\
       &| &\langle(PV \text{ : } EC \rightarrow \text{ Set})*\rangle & \\
    Decl &:= &\text{data } N\; PC \text{ : } (EC \rightarrow)? \text{ Set where} &\text{Declarations}\\
              &&\quad(N \text{ : }  (EC \rightarrow)? TypeExpr \rightarrow N\; Expr*)* &\\
         &| &\text{codata } N\; PC \text{ : } (EC \rightarrow)? \text{ Set where}& \\
     &&\quad(N \text{ : }  (EC \rightarrow)? N\; Expr* \rightarrow TypeExpr)*& \\
    &| &n \; PC \; EC \text{ = } Expr & \\
  \end{array}
\end{align*}
\caption{Syntax for declarations}
\label{syntax-for-declarations}
\end{figure}

The declarations in Figure \ref{syntax-for-declarations} correspond to \(\rho(X:\Gamma\rat*;\vv\sigma;\vv{A}):\Gamma\rat*\) as follows:
\begin{itemize}
\item The first \(N\) is X
\item The other \(N\) will be used later for
\(\alpha_1^{\mu(X:\Gamma\rat *;\vv\sigma;\vv A)},\alpha_2^{\mu(X:\Gamma\rat *;\vv\sigma;\vv A)},\dots\)
in the case of inductive types and
\(\xi_1^{\nu(X:\Gamma\rat *;\vv\sigma;\vv A)},\xi_2^{\nu(X:\Gamma\rat *;\vv\sigma;\vv A)},\dots\)
in the coinductive case
\item The \(TypExpr\) are the \(\vv{A}\)
\item The \(Expr*\) are the \(\vv{\sigma}\)
\item The first \(EC\) is \(\Gamma\)
\item The other \(EC\) stand for \(\Gamma_1,\dots,\Gamma_m\)
\end{itemize}

To parse the abstract syntax we use Megaparsec. The parser generates an
abstract syntax tree, which is given for declarations in Listing
\ref{Abstract Syntax Tree for Declarations}. The field $\mathtt{ty}$\;in $\mathtt{ExprDef}$\;is used later in
typechecking. The parser just fills them in with $\mathtt{Nothing}$. Data and codata
definitions both correspond to $\mathtt{TypeDef}$. The Haskell type $\mathtt{OpenDuctive}$\;contains all the
information for inductive and coinductive types. It corresponds to \(\mu\) and
\(\nu\) in the paper. We use an $\mathtt{OpenDuctive}$\;where the field $\mathtt{inOrCoin}$\;is $\mathtt{IsIn}$
for \(\mu\) and an $\mathtt{OpenDuctive}$\;where the field $\mathtt{inOrCoin}$\;is $\mathtt{IsCoin}$\;for
\(\nu\).  The Haskell type $\mathtt{StrDef}$\;ensures that the sigmas, as and gamma1s have the
same length.  We omit the implementation details for the parser because we
are mainly focused on typechecking.

\begin{listing}[htbp]
\begin{minted}[fontsize=\scriptsize]{haskell}
data Decl = ExprDef { name :: Text
                    , tyParameterCtx :: TyCtx
                    , exprParameterCtx :: Ctx
                    , expr :: Expr
                    , ty :: Maybe Type
                    }
          | TypeDef OpenDuctive
          | Expression Expr

data OpenDuctive = OpenDuctive { nameDuc :: Text
                               , inOrCoin :: InOrCoin
                               , parameterCtx :: TyCtx
                               , gamma :: Ctx
                               , strDefs :: [StrDef]
                               }

data StrDef = StrDef { sigma :: [Expr]
                     , a :: TypeExpr
                     , gamma1 :: Ctx
                     , strName :: Text
                     }
\end{minted}
\caption{\label{Abstract Syntax Tree for Declarations}
Implementation of the abstract syntax of fig. \ref{syntax-for-declarations}}
\end{listing}
\subsection{Expressions}
\label{sec:org09156e6}
The abstract syntax for expressions is given in Figure \ref{syntax-for-expressions}.
We will separate expressions in expressions for terms and expressions for
types.  They are given in $\mathtt{Expr}$\;and $\mathtt{TypeExpr}$\;respectively.

\begin{figure}
 \begin{align*}
   \begin{array}{llll}
     ParInst &:= &\langle TypeExpr(\text{,}TypeExpr)*\rangle &\text{Instantiations for}\\
                                                            &&&\text{paramter contexts}\\
     ExprInst &:= &\text{(}Expr(\text{,}Expr)*\text{)} &\text{Instantiations for}\\
                                                       &&&\text{expression contexts}\\
     Expr &:= &\text{rec } N \; ParInst? \text{ to } TypeExpr \text{ where} &\text{expression}\\
     &&\quad Match*&\\
     &| &\text{corec } TypeExpr \text { to } N \; ParInst? \text{ where}&\\
     &&\quad Match*&\\
     &| &Expr\text{ @ }Expr &\\
     &| &\lozenge&\\
     &| &EV&\\
     &| &n\; ParInst\; ExprInst&\\
     Match &:= &N\; EV* = Expr &\text{match}\\
     TypeExpr &:= &\text{(}EV\text{ : }TypeExpr\text{).}TypeExpr &\text{Type expressions}\\
     &| &TypeExpr\text{ @ }Expr&\\
     &| &\text{Unit} &\\
     &| &TV&\\
     &| &N\; ParInst? &\\
  \end{array}
\end{align*}
\begin{lstlisting}
\end{lstlisting}
\caption{Syntax for expressions}
\label{syntax-for-expressions}
\end{figure}

An $\mathtt{Expr}$\;is either a pattern match $\mathtt{rec}$, a copattern match $\mathtt{corec}$, a
con/destructor, an application $\mathtt{@}$, the only primitive unit expression
$\mathtt{\lozenge}$\;or a variable. With the keyword $\mathtt{rec}$\;we can destruct an
inductive type. We write $\mathtt{N\;ParInst?\;to\;TypeExrp}$, after $\mathtt{rec}$\;to facilitate
typechecking, where $\mathtt{N}$\;is a previously defined inductive type and
$\mathtt{ParInst?}$\;the instantiation of its parameter context. It says we want to
destruct an inductive type to some other type. We have to list all the
constructors that we pattern match on. For each constructor, we write an
expression behind the equal sign, which should be of type $\mathtt{TypeExpr}$\;which
we have given above. In this expression, we can use variables given in the
match expression. The last one is the recursive occurrence.

With the keyword $\mathtt{corec}$\;we can do the same thing to construct a coinductive
type. Here, we have to swap the $\mathtt{N\;ParInst?}$\;and the $\mathtt{TypeExpr}$\;and list the
destructors.

All con/destructors have to be instantiated with all variables in the
parameter contexts of their types. This is done by giving types of the
expected kinds separated by ',' enclosed in $\mathtt{\langle}$\;and $\mathtt{\rangle}$.

The variables are separated into local variables and global variables.
Global variables refer to previously defined expressions. We have to fully
instantiate their parameter contexts and their expression contexts. We can
also apply an expression to another with $\mathtt{@}$. This application is
left-associative. So if we write $\mathtt{t\;@\;s\;@\;v}$\;we mean $\mathtt{(t\;@\;s)\;@\;v}$.

The $\mathtt{typeExpr}$\;is either the unit type $\mathtt{Unit}$, a lambda abstraction on
types, an application, or a variable. In the lambda expression, we have to
give the type of the variable. We apply a type to a term (types can only
depend on terms) with $\mathtt{@}$.  As in the case of term application, this is
also left-associative.  The unit type is the only primitive type
expression.

The generated abstract syntax tree is given in Listing
\ref{abstract-syntax-tree-for-expressions}. The variables for expressions are
separated in $\mathtt{LocalExprVar}$\;and $\mathtt{GlobalExprVar}$. $\mathtt{LocalExprVar}$\;should refer
to variables that are only locally defined i.e. in $\mathtt{Rec}$\;and $\mathtt{Corec}$. We
use de Bruijn indices for them. This facilitates substitution which we will
describe in Section \ref{sec:org813153f}. $\mathtt{GlobalExprVar}$\;refers to variables from
definitions. Here, we just use names. We do the same thing for $\mathtt{LocalTypeVar}$
and $\mathtt{GlobalTypeVar}$. In the abstract syntax tree, we use anonymous
constructors like in the paper. We combine them with the Haskell constructor
$\mathtt{Structor}$. We know from the field $\mathtt{ductive}$\;if it is a constructor or a
destructor. The types in field $\mathtt{parameters}$\;are to fill in the parameter
context of the field $\mathtt{ductive}$. The field $\mathtt{nameStr}$\;in $\mathtt{Constructor}$\;and
$\mathtt{Destructor}$\;are just for printing. We combine $\mathtt{rec}$\;and $\mathtt{corec}$\;to $\mathtt{Iter}$.

\begin{listing}[htbp]
\begin{minted}[fontsize=\scriptsize]{haskell}
data TypeExpr = UnitType
              | TypeExpr :@ Expr
              | LocalTypeVar Int Bool Text
              | Parameter Int Bool Text
              | GlobalTypeVar Text [TypeExpr]
              | Abstr Text TypeExpr TypeExpr
              | Ductive { openDuctive :: OpenDuctive
                        , parametersTyExpr :: [TypeExpr]}

data Expr = UnitExpr
          | LocalExprVar Int Bool Text
          | GlobalExprVar Text [TypeExpr] [Expr]
          | Expr :@: Expr
          | Structor { ductive :: OpenDuctive
                     , parameters :: [TypeExpr]
                     , num :: Int
                     }
          | Iter { ductive :: OpenDuctive
                 , parameters :: [TypeExpr]
                 , motive :: TypeExpr
                 , matches :: [([Text],Expr)]
                 }
\end{minted}
\caption{\label{abstract-syntax-tree-for-expressions}
Implementation of the abstract syntax of fig. \ref{syntax-for-expressions}}
\end{listing}

\section{Substitution}
\label{sec:org813153f}
In the following we will write \(t[s/x]\) for "substitute every free
occurrences of \(x\) in \(t\) by \(s\)". Substitution is done in the module
$\mathtt{Subst.hs}$. We use de Bruijn indexes \cite{de1972lambda} for bound variables to facilitate
substitution. With this method, every bound variable is a number instead of a
string. The number says where the variable is bound. To find the binder of a
variable we go outwards from it and count every binder until we reach the
number of the variable. For example, \(\lambda.\lambda.\lambda.1\) says that the
variable is bound by the second binder (we start counting at zero). This
would be the same as \(\lambda x.\lambda y. \lambda z.y\). This means we never
have to generate fresh names. We just shift the free variables in the term
with which we substitute by one, every time we encounter a binder. This
shifting is done in the module $\mathtt{ShiftFreeVars.hs}$. We also want to be able to
substitute multiple variables simultaneously. If we would just substitute one
term after another we could substitute into a previous term. For example, the
substitution \(x[y/x][z/y]\) would yield \(z\) if we substitute sequential and
\(y\) if we substitute simultaneously.  To make simultaneous substitution
possible every local variable has a boolean flag.  If this flag is set to
true substitution won't substitute for that variable.  So for simultaneous
substitutions, we just set this flag to true for all terms with which we want
to substitute.  Then, we substitute with them.  In the last step, we just have
to set the flags to false in the result.  This setting(marking of the
variables) is done in the module $\mathtt{Mark.hs}$.

\section{Typing Rules}
\label{sec:org11f03b3}
A typing rule says that some expression or declaration is of some type, given
some premises. If we can for every declaration or expression form a tree of
such rules with no open premises, our program typechecks. We have to rewrite
the typing rules of the paper, to get rules which are syntax-directed.
Syntax-directed means we can infer from the syntax alone what we have to check next
i.e. which rule with which premises we have to apply. In the paper, there are
rules containing variables in the premises where their type isn't in the
conclusion. So if we want to typecheck something which is the conclusion of
such a rule we have no way of knowing what these variables are.

We don't need the weakening rules because we can look up a variable in a
context.  So we ignore them in our implementation.

We also rewrite the rules which are already syntax-directed to rules which
work on our syntax.   We will mark semantic differences in the rewritten rules
gray. We use variables \(\Phi,\Phi',\Phi_1,\Phi_2,\dots\) for parameter contexts,
\(\Theta,\Theta',\Theta_1,\Theta_2,\dots\) for type variable contexts and
\(\Gamma,\Gamma',\Gamma_1,\Gamma_2,\dots\) for term variable contexts.
The following judgments forms exist in our system.
\begin{itemize}
\item \(\Phi\mid\Theta\mid\Gamma\vdash\Theta'\) - The type variable context
\(\Theta'\) is well-formed in the combined context \(\Phi\mid\Theta\mid\Gamma\).
\item \(\Phi\mid\Theta\mid\Gamma\vdash\Gamma'\) - The term variable context
\(\Gamma'\) is well-formed in the combined context \(\Phi\mid\Theta\mid\Gamma\).
\item \(\Phi\mid\Theta\mid\Gamma\vdash\Phi'\) - The parameter variable context
\(\Phi'\) is well-formed in the combined context \(\Phi\mid\Theta\mid\Gamma\).
\item \(A\longrightarrow_T^* B\) - The type \(A\) fully evaluates to type \(B\).
\item \(A \equiv_\beta B\) - The type \(A\) is computationally equivalent to type \(B\).
\item \(\Phi\mid\Theta\mid\Gamma\vdash A : \Gamma_2\rat*\) - The type
\(A\) is well-formed in the combined context \(\Phi\mid\Theta\mid\Gamma\) and
can be instantiated with arguments according to context \(\Gamma_2\).
\item \(\Phi\mid\Theta\mid\Gamma\vdash t : \Gamma_2\rat A\) - The term \(t\) is
well-formed in the combined context \(\Phi\mid\Theta\mid\Gamma\) and can be
instantiated with arguments according to context \(\Gamma_2\).  After this
instantiation, it is of type \(A\), where the arguments are substituted in \(A\).
\item \(\Phi \vdash \sigma : \Gamma_1 \triangleright \Gamma_2\) - The context
morphism \(\sigma\) is a well-formed substitution for \(\Gamma_2\) with terms
in context \(\Gamma_1\) in parameter context \(\Phi\).
\end{itemize}
We will write \(\vdash\) for \(\Phi\mid\Theta\mid\Gamma\vdash\) where
\(\Phi\),\(\Theta\), and \(\Gamma\) are arbitrary and aren't referred to by the
right-hand side.

In the module $\mathtt{TypeChecker}$\;we will implement the following rules.  It
defines a monad $\mathtt{TI}$\;which can throw errors and has a reader on the contexts
in which we are typechecking.

\subsection{Context Rules}
\label{sec:orgeb36003}
The rules for valid contexts are already-syntax directed so we just take
them unmodified.
\begin{center}
\AxiomC{}
\UnaryInfC{$\vdash\emptyset$ \TyCtx}
\DisplayProof
\hskip 1.5em
\AxiomC{$\vdash\Theta$ \TyCtx}
\AxiomC{$\vdash\Gamma$ \Ctx}
\BinaryInfC{$\vdash\Theta,X:\Gamma\rat*$ \TyCtx}
\DisplayProof
\vskip 0.5em
\AxiomC{}
\UnaryInfC{$\vdash\emptyset$ \Ctx}
\DisplayProof
\hskip 1.5em
\AxiomC{$\mid\emptyset\mid\Gamma\vdash A:*$}
\UnaryInfC{$\vdash\Gamma,x:A$ \Ctx}
\DisplayProof
\end{center}
In the rules for valid contexts, we ensure that the types in the context can
not depend on \textbf{TyCtx}.  Note however that they can depend on \textbf{ParCtx}.  This
ensures that only strictly positive types are possible.


The order in \textbf{TyCtx} isn't relevant so we can use a map for it. In the code, we
use a list because the names of the variables are the index of their type in
the context. The order of \textbf{Ctx} is relevant because types of later variables
can refer to former variables and application instantiates the first variable
in \textbf{Ctx}. We add a new context for data types. We also need a context for the
parameters. \textbf{Ctx} can contain variables from this context, but not from
\textbf{TyCtx}.

We also need new rules for checking if a parameter context is valid.
\begin{center}
\AxiomC{}
\UnaryInfC{$\vdash\emptyset$ \ParCtx}
\DisplayProof
\hskip 1.5em
\AxiomC{$\vdash\Phi$ \ParCtx}
\AxiomC{$\vdash\Gamma$ \Ctx}
\BinaryInfC{$\vdash\Phi,X:\Gamma\rat*$ \ParCtx}
\DisplayProof
\end{center}
These rules are structurally identical to the rules for \textbf{TyCtx}.  The difference is that \textbf{ParCtx}
and \textbf{TyCtx} are used differently in the other rules, as we have already seen
in the rule for \textbf{Ctx}.

We wirte \(\Theta(X)\rightsquigarrow\Gamma\rat*\), if looking up
the type variable \(X\) in type context \(\Theta\) yields the type \(\Gamma\rat*\). We
add 2 rules for looking up something in a type context. They are:
\begin{center}
  \AxiomC{$\vdash \Theta$ \TyCtx}
  \AxiomC{$\vdash \Gamma$ \Ctx}
  \BinaryInfC{$\Theta,X:\Gamma\rat*(X)\rightsquigarrow\Gamma\rat*$}
  \DisplayProof
  \hskip 1.5em
  \AxiomC{$\vdash \Gamma_1$ \Ctx}
  \AxiomC{$\Theta(X) \rightsquigarrow\Gamma_2\rat*$}
  \BinaryInfC{$\Theta,Y:\Gamma_1\rat*(X)\rightsquigarrow\Gamma_2\rat*$}
  \DisplayProof
\end{center}
Here, \(Y\) and \(X\) are different variables.

The rules for looking up something in a parameter context are principally the
same.
\begin{center}
  \AxiomC{$\vdash \Phi$ \ParCtx}
  \AxiomC{$\vdash \Gamma$ \Ctx}
  \BinaryInfC{$\Phi,X:\Gamma\rat*(X)\rightsquigarrow\Gamma\rat*$}
  \DisplayProof
  \hskip 1.5em
  \AxiomC{$\vdash \Gamma_1$ \Ctx}
  \AxiomC{$\Phi(X) \rightsquigarrow\Gamma_2\rat*$}
  \BinaryInfC{$\Phi,Y:\Gamma_1\rat*(X)\rightsquigarrow\Gamma_2\rat*$}
  \DisplayProof
\end{center}

Respectively the notation \(\Gamma(x)\rightsquigarrow A\) means looking
up the term variable \(x\) in term context \(\Gamma\) yields type \(A\). The
rules for term contexts are:
\begin{center}
  \AxiomC{$\vdash \Gamma$ \Ctx}
  \AxiomC{$\Gamma\vdash A:*$}
  \BinaryInfC{$\Gamma,x:A(x)\rightsquigarrow A$}
  \DisplayProof
  \hskip 1.5em
  \AxiomC{$\Gamma(x) \rightsquigarrow A$}
  \AxiomC{$\Gamma\vdash B:*$}
  \BinaryInfC{$\Gamma,y:B(x)\rightsquigarrow A$}
  \DisplayProof
\end{center}

\subsection{Beta-equivalence}
\label{sec:org991ede0}
Two types are \(\beta\text{-equivalent}\) if they evaluate to the same type. Because
our language is normelizing it is sufficent to fully evaluate both of them
and check it they are \(\alpha\text{-equivalent.}\) So we first need to define rules
which say what full evaluation means. We write \(A \longrightarrow_T^* B\) if
evaluating \(A\) as long as it is possible results in \(B\).

The rules are:
\begin{center}
\AxiomC{$\neg\exists B : A \longrightarrow_T B$}
\UnaryInfC{$A \longrightarrow_T^* A$}
\DisplayProof
\hskip 1.5em
\AxiomC{$A \longrightarrow_T B$}
\AxiomC{$B \longrightarrow_T^* C$}
\BinaryInfC{$A \longrightarrow_T^* C$}
\DisplayProof
\end{center}
\(\longrightarrow_T\) is defined in Section \ref{sec:orgedabc94}.

We can then introduce a new rule for \(\beta\text{-equivalence}\).
\begin{center}
\AxiomC{$A\longrightarrow_T^* A'$}
\AxiomC{$B\longrightarrow_T^* B'$}
\AxiomC{$A'\equiv_\alpha B'$}
\TrinaryInfC{$A\equiv_\beta B$}
\DisplayProof
\end{center}
This rule says if \(A\) evaluates to \(A'\), \(B\) to \(B'\) and \(A'\) and \(B'\) are
\(\alpha\text{-equivalent,}\) then \(A\) and \(B\) are \(\beta\text{-equivalent.}\) In the
implementation \(\equiv_\alpha\) is trivial because we use \emph{de Bruijn
indices}.

We also add some rules to check if two contexts are the same.
   \begin{center}
   \AxiomC{}
   \UnaryInfC{$\emptyset\equiv_\beta\emptyset$}
   \DisplayProof
   \hskip 1.5em
   \AxiomC{$\Gamma_1\equiv_\beta \Gamma_2$}
   \AxiomC{$A\equiv_\beta B$}
   \BinaryInfC{$\Gamma_1,x:A\equiv_\beta\Gamma_2,y:B$}
   \DisplayProof
%   \vskip 0.5em
%   \AxiomC{$\Theta_1\equiv_\beta \Theta_2$}
%   \AxiomC{$\Gamma_1\equiv_\beta \Gamma_2$}
%   \BinaryInfC{$\Theta_1,X:\Gamma_1\rat*\equiv_\beta\Theta_2,X:\Gamma_2\rat*$}
%   \DisplayProof
   \end{center}

\subsection{Unit Type}
\label{sec:org1e37d83}
The paper defines one rule for the unit type and one for the unit value.
These are.
\begin{center}
  \AxiomC{}
  \RightLabel{\textbf{($\top$-I)}}
  \UnaryInfC{$\vdash\top:*$}
  \DisplayProof
  \hskip 1.5em
  \AxiomC{}
  \topI{$\vdash\lozenge:\top$}
  \DisplayProof
\end{center}
The first rule says that the type \(\top\) has always an empty context.  The
second rule says its value \(\lozenge\) is always of type \(\top\). These rules
get rewritten to.
\begin{center}
  \AxiomC{}
  \RightLabel{\textbf{(Unit-I)}}
  \UnaryInfC{\graybox{$\Phi\mid\Theta\mid\Gamma$}$\vdash$Unit:$*$}
  \DisplayProof
  \hskip 1.5em
  \AxiomC{}
  \topI{\graybox{$\Phi\mid\Theta\mid\Gamma$}$\vdash\lozenge$:Unit}
  \DisplayProof
\end{center}
We change the syntax "\(\top\)" to "Unit" and add the contexts \(\Phi\),
\(\Theta\), \(\Gamma\). We will do this for every rule which has empty contexts
to subsume the weakening rules of the paper. The unit term always has the
unit type as its type.

\subsection{Variable Lookup}
\label{sec:orgfcacb0f}
We have three kinds of variables we can lookup. They are type variables,
term variables, and parameters.  The paper already has rules for the type and
term variables,  which we need to rewrite.  We add a new rule for looking up
a parameter.

The rule:
 \begin{prooftree}
  \AxiomC{$\vdash \Theta$ \TyCtx}
  \AxiomC{$\vdash \Gamma$ \Ctx}
  \TyVarI{$\Theta,X:\Gamma\rat*\mid\emptyset\vdash X : \Gamma \rat *$}
\end{prooftree}
gets rewritten to:
\begin{prooftree}
  \AxiomC{\graybox{$\Theta(X)\rightsquigarrow\Gamma\rat*$}}
  \AxiomC{\graybox{$\vdash \Gamma_1$ \Ctx}}
  \TyVarI{\graybox{$\Phi$}$\mid\Theta\mid$\graybox{$\Gamma_1$}$\vdash X : \Gamma \rat *$}
\end{prooftree}
The rule:
\begin{center}
  \AxiomC{$\Gamma\vdash A:*$}
  \RightLabel{\textbf{(Proj)}}
  \UnaryInfC{$\Gamma,x:A\vdash x:A$}
  \DisplayProof
\end{center}
gets rewritten to:
\begin{center}
  \AxiomC{\graybox{$\Gamma(x)\rightsquigarrow A$}}
  \RightLabel{\textbf{(Proj)}}
  \UnaryInfC{\graybox{$\Phi\mid\Theta\mid$}$\Gamma\vdash x:A$}
  \DisplayProof
\end{center}
The rule for looking something up in the parameter context is:
\begin{prooftree}
  \AxiomC{$\Phi(X)\rightsquigarrow\Gamma\rat*$}
  \AxiomC{$\vdash \Gamma_1$ \Ctx}
  \TyVarI{$\Phi\mid\Theta\mid$$\Gamma_1$$\vdash X : \Gamma \rat *$}
\end{prooftree}

In the rules from the paper, we can only infer the type or kind of the last variable in the
context.  In our rules, we just look up the variable in the context.  These
rules can check the same thing if we take the weakening rules into account.
With them, we can just weaken the context until we get to the desired
variable.

\subsection{Type and Expression Instantiation}
\label{sec:org461d8fa}
We can instantiate types and terms.  The rule:
\begin{prooftree}
  \AxiomC{$\Theta\mid\Gamma_1\vdash A:(x:B,\Gamma_2)\rat*$}
  \AxiomC{$\Gamma_1\vdash t:B$}
  \TyInst{$\Theta\mid\Gamma_1\vdash A@t:\Gamma_2[t/x]\rat*$}
\end{prooftree}
for instantiating types gets rewritten to:
 \begin{prooftree}
  \AxiomC{\graybox{$\Phi$}$\mid\Theta\mid\Gamma_1\vdash A:(x:B,\Gamma_2)\rat*$}
  \AxiomC{\graybox{$\Phi\mid\Theta$}$\mid\Gamma_1\vdash t:$\graybox{$B'$}}
  \AxiomC{\graybox{$B\equiv_\beta B'$}}
  \TyInstTrinary{\graybox{$\Phi$}$\mid\Theta\mid\Gamma_1\vdash A@t:\Gamma_2[t/x]\rat*$}
\end{prooftree}
For this rule, we have to check if \(t\) has the expected type for the first
variable in the context of \(A\).  In our version, we just infer the type for \(A\) and \(t\).
Then, we check if the first variable in the context is \(\beta\text{-equal}\) to the type
of \(t\).  If that isn't the case typechecking fails.  Otherwise, we just
substitute in the remaining context.

We also have a rule to instantiate terms.  This rule:
\begin{center}
  \AxiomC{$\Gamma_1\vdash t:(x:A,\Gamma_2)\rat B$}
  \AxiomC{$\Gamma_1\vdash s:A$}
  \RightLabel{\textbf{(Inst)}}
  \BinaryInfC{$\Gamma_1\vdash t@s:\Gamma_2[s/x]\rat B[s/x]$}
  \DisplayProof
\end{center}
gets rewritten to:
\begin{center}
  \AxiomC{\graybox{$\Phi\mid\Theta$}$\mid\Gamma_1\vdash t:(x:A,\Gamma_2)\rat B$}
  \AxiomC{\graybox{$\Phi\mid\Theta$}$\mid\Gamma_1\vdash s:$\graybox{$A'$}}
  \AxiomC{\graybox{$A\equiv_\beta A'$}}
  \RightLabel{\textbf{(Inst)}}
  \TrinaryInfC{\graybox{$\Phi\mid\Theta$}$\mid\Gamma_1\vdash t@s:\Gamma_2[s/x]\rat B[s/x]$}
  \DisplayProof
\end{center}
These rules are similar to the rule for type instantiation.  Here, we have to
check (or infer) a term instead of a type.  We also have to substitute \(s\) in
the result type of \(t\) (in the case of types it's always \(*\), which obviously
has no free variables).

\subsection{Parameter Abstraction}
\label{sec:org629f9ea}
The rule:
\begin{center}
  \AxiomC{$\Theta\mid\Gamma_1,x:A\vdash B:\Gamma_2\rat*$}
  \ParamAbstr{$\Theta\mid\Gamma_1\vdash(x).B:(x:A,\Gamma_2)\rat*$}
  \DisplayProof
\end{center}
gets rewritten to:
\begin{center}
  \AxiomC{\graybox{$\Phi$}$\mid\Theta\mid\Gamma_1,x:A\vdash B:\Gamma_2\rat*$}
  \ParamAbstr{\graybox{$\Phi$}$\mid\Theta\mid\Gamma_1\vdash(x$\graybox{$:A$}$).B:(x:A,\Gamma_2)\rat*$}
  \DisplayProof
\end{center}
Here, we just add the argument of the lambda to the expression context.  Then
we check the body of the lambda.  In the syntax-directed version we have to
annotate the variable with its type, so we know which type we have to add to
the context.

\subsection{(Co)Inductive Types}
\label{sec:orgf0f8c72}
We have to separate the rule:
\begin{prooftree}
\AxiomC{$\sigma_k:\Gamma_k\triangleright\Gamma$}
\AxiomC{$\Theta,X:\Gamma\rat*\mid\Gamma_k\vdash A_k:*$}
\FPTy
\BinaryInfC{$\Theta \mid \emptyset \vdash \rho(X : \Gamma \rat *;\vv{\sigma};\vv{A}):\Gamma\rat *$}
\end{prooftree}
into multiple rules.  First, we need rules to check the definitions of
(co)inductive types.  These are:
\begin{prooftree}
\AxiomC{$\sigma_k:\Gamma_k\triangleright\Gamma$}
\AxiomC{\graybox{$\Phi$}$\mid X:\Gamma\rat*\mid\Gamma_k\vdash A_k:*$}
\AxiomC{\graybox{$\vdash \phi$ \ParCtx}}
\FPTy
\TrinaryInfC{$\vdash$ data X$\langle\Phi\rangle$ $\Gamma \rat $ Set where; $\vv{Constr_k : \Gamma_k\rat A_k\rat X \sigma_k}$}
\end{prooftree}
\begin{prooftree}
\AxiomC{$\sigma_k:\Gamma_k\triangleright\Gamma$}
\AxiomC{\graybox{$\Phi$}$\mid X:\Gamma\rat*\mid\Gamma_k\vdash A_k:*$}
\AxiomC{\graybox{$\vdash \phi$ \ParCtx}}
\FPTy
\TrinaryInfC{$\vdash$ codata X$\langle\Phi\rangle$ : $\Gamma \rat$ Set where; $\vv{Destr_k : \Gamma_k \rat  X \sigma_k \rat  A_k}$}
\end{prooftree}
Because we only allow top-level definitions of (co)inductive types our rules
have empty contexts.  We first have to check if \(\sigma_k\) is  a context
morphism from \(\Gamma_k\) to \(\Gamma\).  This basically means that the terms
in \(\sigma_k\) are of the types in \(\Gamma\), if we check them in \(\Gamma_k\).
After that, we have to check if the \(\vv{A}\) (the arguments where we can
have a recursive occurrence) are of kind \(*\). Because this is a top-level
definition the context \(\phi\) is provided explicitly in the definition. So we
have to check if it is valid. We will now have to rewrite the rules for
context morphisms. Here, we just add the parameter context to the rules of
the paper.
\begin{center}
\AxiomC{}
\UnaryInfC{\graybox{$\Phi\vdash$}$() : \Gamma_1 \triangleright \emptyset$}
\DisplayProof
\hskip 1.5em
\AxiomC{\graybox{$\Phi\vdash$}$\sigma : \Gamma_1 \triangleright \Gamma_2$}
\AxiomC{\graybox{$\Phi\mid$}$\Gamma_1\vdash t : A[\sigma]$}
\BinaryInfC{\graybox{$\Phi\vdash$}$(\sigma,t):\Gamma_1\triangleright(\Gamma_2,x:A)$}
\DisplayProof
\end{center}
We also need a rule for the cases in which we are using these defined
variables.  This rule is:
\begin{prooftree}
\AxiomC{$\Phi\mid\Theta\mid\Gamma'\vdash \vv{A}:\Gamma_i \rat *$}
\UnaryInfC{$\Phi\mid\Theta\mid\Gamma'\vdash X\langle\vv{A}\rangle : \Gamma[\vv{A}]\rat *$}
\end{prooftree}
Here, \(X\) is a data or codata definition.  The parser can decide if a variable
is such a definition or a local definition. Because we are typechecking
on the abstract syntax tree we also know \(\Gamma\) and \(\Phi'\). \(\Gamma\) is
just the context from the definition and \(\Phi\) is the parameter context.
Because we already typed checked this definition we just have to check if
the types given for the parameters have the right kind.  Then, we substitute
these parameters in its type.  We will now give the rules for checking if a
list of parameters matches a parameter context.
\begin{center}
\AxiomC{}
\UnaryInfC{$\Phi\mid\Theta\mid\Gamma\vdash () : ()$}
\DisplayProof
\hskip 1.5em
\AxiomC{$\Phi\mid\Theta\mid\Gamma\vdash A : \Gamma'\rat*$}
\AxiomC{$\Phi\mid\Theta\mid\Gamma\vdash \vv{A} : \Phi'[A/X]$}
\BinaryInfC{$\Phi\mid\Theta\mid\Gamma\vdash A,\vv{A} : (X:\Gamma'\rat*,\Phi'$)}
\DisplayProof
\end{center}
We just check every variable for the kinds in \(\Phi'\) one after the other.
We also have to substitute the type into the context because kinds in
a parameter context can depend on previously defined variables in this context.

\subsection{Constructor and Destructor}
\label{sec:orgadc6cc9}
The rule for constructors:
\begin{center}
  \AxiomC{$\mu(X:\Gamma\rat*;\vv{\sigma};\vv{A}):\Gamma\rat*$}
  \AxiomC{$1\leq k\leq\mid\vv{A}\mid$}
  \IndIBinary{$\vdash\alpha_k^{\mu(X:\Gamma\rat*;\vv{\sigma};\vv{A})}:(\Gamma_k,y:A_k[\mu/X])\rat\mu@\sigma_k$}
  \DisplayProof
\end{center}
gets rewritten to:
\begin{center}
  \AxiomC{\graybox{$\Phi\mid\Theta\mid\Gamma\vdash \vv{B} : \Phi'$}}
  \IndI{\graybox{$\Phi\mid\Theta\mid\Gamma$}$\vdash$Constr\graybox{$\langle\vv{B}\rangle$}$:(\Gamma_k\graybox{$\graybox{$[\vv{B}]$}$},y:A_k[\mu/X]\graybox{$\graybox{$[\vv{B}]$}$})\rat\mu@\sigma_k\graybox{$\graybox{$[\vv{B}]$}$}$}
  \DisplayProof
\end{center}
The rule for destructors:
\begin{center}
  \AxiomC{$\nu(X:\Gamma\rat*;\vv{\sigma};\vv{A}):\Gamma\rat*$}
  \AxiomC{$1\leq k\leq\mid\vv{A}\mid$}
  \RightLabel{\textbf{(Coind-E)}}
  \BinaryInfC{$\vdash\xi_k^{\nu(X;\Gamma\rat*;\vv{\sigma};\vv{A})}:(\Gamma_k,y:\nu@\sigma_k)\rat
    A_k[\nu/X]$}
  \DisplayProof
\end{center}
gets rewritten to:
\begin{center}
  \AxiomC{\graybox{$\Phi\mid\Theta\mid\Gamma\vdash \vv{B} : \Phi'$}}
  \RightLabel{\textbf{(Coind-E)}}
  \UnaryInfC{\graybox{$\Phi\mid\Theta\mid\Gamma$}$\vdash$Destr\graybox{$\langle\vv{B}\rangle$}$:(\Gamma_k$\graybox{$[\vv{B}]$}$,y:\nu@\sigma_k)$\graybox{$[\vv{B}]$}$\rat
    A_k[\nu/X]$\graybox{$[\vv{B}]$}$$}
  \DisplayProof
\end{center}
In the paper de/constructors are anonymous.  They come together with their
type. Therefore, we have to check if this type is valid. Constructors
construct their type. So their output value is their type \(\mu\) applied to
the context morphism \(\sigma_k\), where \(k\) is the number of the constructor.
They become as input the context \(\Gamma_k\), which is implicit in the paper,
and a value of type \(A_k[\mu/X]\), which is the type, which can contain the
recursive occurrence. Destructors are destructing their type so we get their
type \(\nu\) applied to \(\sigma_k\) as input and \(A_k[\nu/X]\) as output.

In our rules, in contrast to the paper, the de/constructors refer to some
type which we have already type-checked. We just have to check the
parameters. Every term we need is in the Haskell representation of the
de/constructor. The de/constructor has the type which we have defined in the
data definition. We just substitute the type itself for the free variable.
At last, we need to substitute the parameters for the respective variables.

\subsection{Recursion and Corecursion}
\label{sec:org8d537fa}
The rule:
\begin{center}
  \AxiomC{$\vdash C:\Gamma\rat*$}
  \AxiomC{$\Delta,\Gamma_k,y_k:A_k[C/X]\vdash g_k:(C@\sigma_k)$}
  \AxiomC{$\forall k=1,\dots,n$}
  \RightLabel{\textbf{(Ind-E)}}
  \TrinaryInfC{$\Delta\vdash$ rec
    $\vv{(\Gamma_k,y_k).g_k}:(\Gamma,y:\mu@id_\Gamma)\rat C@id_\Gamma$}
  \DisplayProof
\end{center}
gets rewritten to:
\begin{scprooftree}{0.95}
  \AxiomC{$\vdash C:\Gamma\rat*$}
  \AxiomC{\graybox{$\vdash\Gamma\equiv_\beta \Gamma'[\vv{D}]$}}
  \noLine
  \UnaryInfC{\graybox{$\vv{\vdash B_k\equiv_\beta(C@\sigma_k[\vv{D}])}$}}
  \AxiomC{\graybox{$\Phi\mid\Theta\mid\Delta\vdash \vv{D}:\Phi'$}}
  \noLine
  \UnaryInfC{$\vv{$\graybox{$\Phi\mid\emptyset\mid$}$\Delta,\Gamma_k$\graybox{$[\vv{D}]$}$,y_k:A_k$\graybox{$[\vv{D}]$}$[C/X]\vdash g_k:\text{\graybox{$B_k$}}}$}
  \RightLabel{\textbf{(Ind-E)}}
  \TrinaryInfC{\graybox{$\Phi\mid\Theta\mid$}$\Delta\vdash$ rec \graybox{$\mu\langle\vv{D}\rangle$ to C};
    $\vv{\text{Constr}_k\vv{x_k}\text{ } y_k = g_k}:(\Gamma,y:\mu$\graybox{$[\vv{D}]$}$@id_\Gamma)\rat C@id_\Gamma$}
 \end{scprooftree}

We are recursing over some previously inductively defined type \(\mu\) to some
type \(C\).  These types must have the same context.  Recursing is done by
Listing each constructor with the result, which the whole expression should
have if we apply it to this constructor.  This result can refer to the
arguments of the constructor via the variables \(\vv{x_k},y_k\).  The type
must be the result type \(C\) applied to the \(\sigma_k\) of this constructor.
In the syntax-directed version, we also have to check the parameters.  We
check if the types match by inferring them and compare them on beta
equality.

We have a similar rule for corecursion.  It:
\begin{center}
  \AxiomC{$\vdash C:\Gamma\rat*$}
  \AxiomC{$\Delta,\Gamma_k,y_k:(C@\sigma_k)\vdash g_k:A_k[C/X]$}
  \AxiomC{$\forall k=1,\dots,n$}
  \RightLabel{\textbf{(Coind-I)}}
  \TrinaryInfC{$\Delta\vdash$ corec
    $\vv{(\Gamma_k,y_k).g_k}:(\Gamma,y:C@id_\Gamma)\rat \nu@id_\Gamma$}
  \DisplayProof
\end{center}
gets rewritten to:
\begin{scprooftree}{0.95}
  \AxiomC{$\vdash C:\Gamma\rat*$}
  \AxiomC{\graybox{$\vdash\Gamma\equiv_\beta \Gamma'[\vv{D}]$}}
  \noLine
  \UnaryInfC{\graybox{$\vv{\vdash B_k\equiv_\beta A_k[\vv{D}][C/X]}$}}
  \AxiomC{\graybox{$\Phi\mid\Theta\mid\Delta\vdash \vv{D}:\Phi'$}}
  \noLine
  \UnaryInfC{$\vv{$\graybox{$\Phi\mid\emptyset\mid$}$\Delta,\Gamma_k$\graybox{$[\vv{D}]$}$,y_k:(C@\sigma_k$\graybox{$[\vv{D}]$}$)\vdash g_k:\text{\graybox{$B_k$}}}$}
  \RightLabel{\textbf{(Coind-I)}}
  \TrinaryInfC{\graybox{$\Phi\mid\Theta\mid$}$\Delta\vdash$ corec \graybox{C to $\nu\langle\vv{D}\rangle$};
    $\vv{\text{Destr}_k\vv{x_k}\text{ } y_k = g_k}:(\Gamma,y:C@id_\Gamma)\rat \nu$\graybox{$[\vv{D}]$}$@id_\Gamma$}
 \end{scprooftree}

A corecursion produces a coinductive type \(\nu\).  We have to give it a type
\(C\) and list the destructors together with the expression they should be
destructed to. In this way, we get the analogue of the syntax-directed rule
for recursion.

\section{Evaluation}
\label{sec:orgedabc94}
There are two kinds of reduction steps in this system, which we have implemented
in $\mathtt{Eval.hs}$. Will give the formal definition in the following.

The first is a reduction on the type level (written $\mathtt{\longrightarrow\_T}$). It is defined
as follows:
\begin{align*}
  ((x).A) @ t \longrightarrow_T A[t/x]\\
\end{align*}
It is standard beta reduction. If we apply a lambda \((x).A)\) to a term \(t\) we
substitute this term for the binding variable \(x\) in the body. This body is
then the result of the reduction.

The other is the reduction on the term level (written $\mathtt{\succ}$). To define this
reduction, we need a action on types (written \(\widehat{C}(A)\)) and terms
(written \(\widehat{C}(t)\)), where the following holds.
\begin{prooftree}
  \AxiomC{$X : \Gamma_1\rat*\mid\Gamma_2'\vdash C:\Gamma_2\rat*$}
  \AxiomC{$\Gamma_1, x:A\vdash t:B$}
  \BinaryInfC{$\Gamma_2',\Gamma_2,x:\widehat{C}(A)\vdash\widehat{C}(t):\widehat{C}(B)$}
\end{prooftree}
Here, we have a type \(C\) with a free type variable \(X\) and a term \(t\) of type
\(B\) with a free term variable \(x\) of type \(A\). If we use the action of this
type on \(t\) we get a term with a type of this action on \(B\). This term
contains a free term variable \(x\) of type \(\widehat{C}(A)\), the action applied to \(A\). The
type action is implemented in the module $\mathtt{TypeAction.hs}$. Both the type
action and the evaluation are done in the $\mathtt{Eval}$\;monad. This monad has access
to the previously defined declarations. We will now define the type action.

\begin{definition}
  \text{Let} $n \in \mathbb{N}$ \text{and} $1 \leq i \leq n$.
  \text{Let:}
  \begin{align*}
    X_1 : \Gamma_1 \rat \ast,\ldots,X_n : \Gamma_n \rat \ast\ \mid\ \Gamma' \vdash C : \Gamma \rat \ast \\
    \Gamma_i \vdash A_i : \ast \\
    \Gamma_i \vdash B_i : \ast \\
    \Gamma_i, x : A_i \vdash t_i : B_i
  \end{align*}
  \text{Then, we define the type action on terms inductively over $C$.}
  \begin{align*}
    \begin{array}{ll}
      \widehat{C}(\vv{t},t_{n+1}) = \widehat{C}(\vv{t})
      &\text{for \textbf{(TyVarWeak)}}\\
      \widehat{X_i}(\vv{t})=t_i\\
      \widehat{C'@s}(\vv{t})=\widehat{C'}(\vv{t})[s/y],
      &\text{for }\Theta\mid\Gamma'\vdash C':(y,\Gamma)\rat*\\
      \widehat{(y).C'}(\vv{t})=\widehat{C'}(\vv{t}),
      &\text{for }\Theta\mid(\Gamma',y)\vdash C':\Gamma\rat*\\
      \widehat{\mu(Y:\Gamma\rat*;\vv{\sigma};\vv{D})} =\text{rec}^{R_A}\vv{(\Delta_k,x).g_k}@\id{\Gamma}@x
      &\text{for } \Theta,Y:\Gamma\rat*\mid\Delta_k\vdash D_k:*\\
      \quad\text{with } g_k = \alpha_k^{R_B}@\id{\Delta_k}@\left(\widehat{D_k}(\vv{t},x)\right)\\
      \quad\text{and } R_A=\mu(Y:\Gamma\rat*;\vv{\sigma};\vv{D}[\vv{(\Gamma_i).A}/\vv{X}])\\
      \quad\text{and } R_B=\mu(Y:\Gamma\rat*;\vv{\sigma};\vv{D}[\vv{(\Gamma_i).B}/\vv{X}])\\
      \widehat{\nu(Y:\Gamma\rat*;\vv{\sigma};\vv{D})} =\text{corec}^{R_B}\vv{(\Delta_k,x).g_k}@\id{\Gamma}@x
      &\text{for } \Theta,Y:\Gamma\rat*\mid\Delta_k\vdash D_k:*\\
      \quad\text{with } g_k = \widehat{D_k}(\vv{t},x)[(\xi_k^{R_A}@\id{\Delta_k}@x)/x]\\
      \quad\text{and } R_A=\mu(Y:\Gamma\rat*;\vv{\sigma};\vv{D}[\vv{(\Gamma_i).A}/\vv{X}])\\
      \quad\text{and } R_B=\mu(Y:\Gamma\rat*;\vv{\sigma};\vv{D}[\vv{(\Gamma_i).B}/\vv{X}])\\
    \end{array}
  \end{align*}
  \text{And the type action on types as follows:}
  \begin{equation*}
    \widehat{C}(\vv{A})=C[\vv{(\Gamma_i).A}/\vv{X}]@\id{\Gamma}
  \end{equation*}
\end{definition}

The type action generates a term with a free variable \(x\).  In the type of
this term, we have changed all the free variables to the types of \(\vv{t}\).
We will show the proof in appendix \ref{sec:orgff507b7}.

The reduction on terms  is subdivided into a reduction on recursion and one
on corecursion. Here, \(\sigma_k \bullet\tau\) is a context morphism, where we
first substitute with \(\tau\) and then with \(\sigma_k\).

The reduction on recursion is defined as follows:
\begin{align*}
  \rec \vv{(\Gamma_k,y_k).g_k}@(\sigma_k\bullet\tau)@(\alpha_k@\tau@u)\succ g_k\left[ \widehat{A_k}(\rec\vv{(\Gamma_k,y_k).g_k}@\id{\Gamma}@x)/y_k \right][\tau,u]\\
\end{align*}
If we apply a recursion \(\rec \vv{(\Gamma_k,y_k).g_k}\) to this context
morphism and a constructor \(\alpha_k@\tau@u\), which is fully applied, we
lookup the case for this constructor. In this case, we substitute \(\tau\) for
the variables from \(\Gamma_k\) and \(u\), where we apply the recursion to all
recursive occurrences, for \(y_k\). For this application, we need the type
action. So a recursion is destructing an inductive type and all its recursive
occurrences to another type, while we use different cases for the different
constructors of the type.

On the other hand, corecursion is constructing a coinductive type. It is defined
as follows:
\begin{align*}
  \xi_k@\tau@(\corec \vv{(\Gamma_k,y_k).g_k}@(\sigma_k\bullet\tau)@u)\succ \widehat{A_k}(\corec\vv{(\Gamma_k,y_k).g_k}@\id{\Gamma}@x)[g_k/x] [\tau,u]
\end{align*}
If we apply a destructor together with its arguments for its context
\(\xi_k@\tau\) on such a construction \((\corec
   \vv{(\Gamma_k,y_k).g_k}@(\sigma_k\bullet\tau)@u)\), we are taking the case of
this destructor. In this case, we are applying the corecursion to all
recursive occurrences. The context morphisms \(\tau\) and \(u\) are substituted
as in the case of recursion.

\chapter{Examples}
\label{sec:orgd840521}

In this Section, we reiterate the example types from the paper, but we will use the
syntax which we introduced in \ref{sec:org153f003}.  We will also show some functions
on these types.  On some of them, we will show the reduction steps in detail.

\section{Terminal and Initial Object}
\label{sec:org1cd1678}

The terminal object is a type that has exactly one value. In category theory,
every object in the category has a unique morphism to the terminal object. We
define it as a coinductive type $\mathtt{Terminal}$\;with no destructors. To get a
terminal value we use corecursion on the unit type, which is the first-class
terminal object.
\begin{lstlisting}
codata Terminal : Set where
terminal = corec Unit to Terminal where @ $\lozenge$
\end{lstlisting}
Contrary to the definition in the paper there is no destructor $\mathtt{Terminal}$.
In the paper definitions of coinductive or inductive types need at least one
de/constructor.  Therefore, our definition wouldn't work.

The initial object is a type that has no values. In category theory it is
the object which has a unique morphism to every other object in the category.
We define it inductively as $\mathtt{Intial}$\;with no constructor. In the paper, it is
defined with one constructor which want's one value of the same
type. We can't have a constructor term of type $\mathtt{Intial}$, because to get one we already need
one. Our way of defining it is shorter and more clear. We can't construct a
value of this type because we have no constructors. If we could get something
of type $\mathtt{Intial}$, we could generate a value of arbitrary type $\mathtt{C}$\;with the
following term $\mathtt{exfalsum}$.

\begin{lstlisting}
data Initial : Set where
exfalsum$\langle$C : Set$\rangle$ = rec Initial to C where
\end{lstlisting}



\section{Natural Numbers and Extended Naturals}
\label{sec:org2792671}

We use the approach of Peano to define natural numbers.  Therefore, we use
the inductive type $\mathtt{Nat}$\;with the constructors $\mathtt{Zero}$\;and $\mathtt{Suc}$. Every
constructor has to have an argument, which can contain a recursive
occurrence. Every Type $\mathtt{A}$\;is isomorphic to the function type
$\mathtt{Terminal\;\rat\;A}$. So we use $\mathtt{Terminal}$\;for this occurrence. $\mathtt{Suc}$\;is
the successor. So the meaning of $\mathtt{Suc\;n}$\;is \(n+1\).
\begin{lstlisting}
data Nat : Set where
   Zero : Terminal $\rat$ Nat
   Suc : Nat $\rat$ Nat
zero = Zero @ $\lozenge$
one = Suc @ zero
\end{lstlisting}
We use the identity function on $\mathtt{Nat}$\;to show that reduction works. This
function is defined using recursion.
\begin{lstlisting}
id = rec Nat to Nat where
       Zero u = Zero @ u
       Succ n = Succ @ n
\end{lstlisting}

We use it on one to see all cases.
\begin{lstlisting}
id @ one = id @ (Succ @ zero)
         $\succ$ Succ @ n[$\widehat{X}$(id @ x)/n] [zero]
         = Succ @ $\widehat{X}$(id @ x) [zero]
         = Succ @ (id @ x)[zero]
         = Succ @ (id @ zero)
         = Succ @ (id @ (Zero @ $\lozenge$))
         $\succ$ Succ @ (Zero @ u[$\widehat{\text{Unit}}$(id @ x)/u][$\lozenge$])
         = Succ @ (Zero @ u[$\widehat{\text{Unit}}$(id @ x)/u][$\lozenge$])
         = Succ @ (Zero @ $\widehat{\text{Unit}}$(id @ x)[$\lozenge$])
         = Succ @ (Zero @ x)[$\lozenge$]
         = Succ @ (Zero @ x) = Succ @ zero = one
\end{lstlisting}
As expected the identity recursion applied to one gives back one.

We will now define the extended naturals, which are also called co-natural numbers.
They are natural numbers with an additional value, infinity. We define it
coinductively with the predecessor as its only destructor. The predecessor is
either not defined or another natural number. We use the type $\mathtt{Maybe}$\;to
describe something which is either present (the constructor $\mathtt{Just}$) or
absent(the constructor $\mathtt{Nothing}$). We can define the successor as a
corecursion. The predecessor of the successor of $\mathtt{x}$\;is just $\mathtt{x}$. So the only
case of $\mathtt{corec}$\;returns a $\mathtt{Just\;x}$\;(remember $\mathtt{Prec}$\;returns a $\mathtt{Maybe\langle\;Conat\rangle}$
not a $\mathtt{Conat}$).
\begin{lstlisting}
data Maybe$\langle$A : Set$\rangle$ : Set where
  Nothing : Unit $\rat$ Maybe
  Just : A $\rat$ Maybe
nothing$\langle$A$\rangle$ = Nothing$\langle$A$\rangle$ @ $\lozenge$
codata Conat : Set where
  Prec : Conat $\rat$ Maybe$\langle$Conat$\rangle$
succ = corec Conat to Conat where
         Prec x = Just$\langle$Conat$\rangle$ @ x
\end{lstlisting}
We now define the values zero and infinity.

\begin{lstlisting}
zero = (corec Unit to Conat where
          {Prev x = nothing$\langle$Unit$\rangle$}) @ $\lozenge$
infinity = (corec Unit to Conat where
              {Prev x = Just$\langle$Conat$\rangle$ @ x}) @ $\lozenge$
\end{lstlisting}

For $\mathtt{zero}$\;the predecessor is absent, there is no predecessor of 0 in the
natural numbers, so we give back $\mathtt{Nothing}$.  We then have to apply the
$\mathtt{corec}$\;to $\mathtt{\lozenge}$\;to get the value.  The predecessor of $\mathtt{infinity}$\;should also
be $\mathtt{infinity}$.  We apply the $\mathtt{corec}$\;to another $\mathtt{Conat}$, so the $\mathtt{x}$\;is also a
$\mathtt{Conat}$.  We will now see that the predecessor on these values gives back the
right value.

{\scriptsize
\begin{flalign*}
  \Prev \apply \zero &\succ \widehat{\Maybe\langle X\rangle} \left(\underbrace{
     \begin{subarray}{l}
       \corecT{\Unit}{\Conat}\\
       \quad\{\;\Prev\; x = \nothing\langle\Unit\rangle\;\}
     \end{subarray}}_{t_1}\apply x\right) [\nothing\langle\Unit\rangle/x][\lozenge]&\\
              &= \recT{\Maybe\langle\Unit\rangle}{\Maybe\langle\Conat\rangle}&\\
              &\quad\quad \{ \Nothing\; u = \Nothing\langle\Conat\rangle \apply \widehat{\Unit}(t_1,u)&\\
              &\quad\quad  \;\Just\; c = \Just\langle\Conat\rangle \apply \widehat{X}(t_1,c)\} \apply x [\nothing\langle\Unit\rangle/x][\lozenge]&\\
              &= \underbrace{\begin{subarray}{l}
                \recT{\Maybe\langle\Unit\rangle}{\Maybe\langle\Conat\rangle} \\
                \quad \{\; \Nothing\; u = \Nothing\langle\Conat\rangle \apply u \\
                \quad \;\;\Just\; c = \Just\langle\Conat\rangle \apply t_1 \}
               \end{subarray}}_{t_2} \apply \nothing\langle\Unit\rangle&\\
              &\succ \Nothing\langle\Conat\rangle \apply u[\widehat{\Unit}(t_2 \apply x)/u][\lozenge]&\\
              &= \Nothing\langle\Conat\rangle \apply u[x/u][\lozenge]&\\
              &= \Nothing\langle\Conat\rangle \apply \lozenge&
\end{flalign*}
\begin{flalign*}
  \Prev \apply \infinity &\succ \widehat{\Maybe\langle X\rangle} \left(\underbrace{
     \begin{subarray}{l}
       \corecT{\Unit}{\Conat}\\
       \quad\{\; \Prev\; x = \Just\langle\Unit\rangle \apply x\; \}
     \end{subarray}}_{t_1} \apply x \right) [\Just\langle\Unit\rangle\apply/x][\lozenge]&\\
                  &= \recT{\Maybe\langle\Unit\rangle}{\Maybe\langle\Conat\rangle}&\\
                  &\quad\quad \{\; \Nothing\; u = \Nothing\langle\Conat\rangle \apply \widehat{\Unit}(t_1,u)&\\
                  &\quad\quad \;\;\Just\; c = \Just\langle\Conat\rangle \apply \widehat{X}(t_1,c)\;\} \apply x [\Just\langle\Unit\rangle\apply/x][\lozenge]&\\
                  &= \underbrace{\begin{subarray}{l}
                    \recT{\Maybe\langle\Unit\rangle}{\Maybe\langle\Conat\rangle} \\
                    \quad \{\; \Nothing\; u = \Nothing\langle\Conat\rangle \apply u \\
                    \quad \;\;\Just\; c = \Just\langle\Conat\rangle \apply t_1\; \}
                   \end{subarray}}_{t_2} \apply \Just\langle\Unit\rangle\apply&\\
                  &\succ \Just\langle\Conat\rangle \apply t_1[\widehat{\Unit}(t_2 \apply x)/x][\lozenge]&\\
                  &= \Just\langle\Conat\rangle \apply t_1[x/x][\lozenge]&\\
                  &= \Just\langle\Conat\rangle \apply \infinity&
\end{flalign*}}

\section{Binary Product and Coproduct}
\label{sec:orgaed1469}

The product is defined as a coinductive type which has two destructors. The
first destructor gives back the first element and the second destructor the
second. To use this type, the types A and B have to be instantiated to
concrete types. We don't have proper type polymorphism in our language, i.e.
we can't write functions which abstract over types, but we have parameterized
types which must be instantiated. We also define a pair expression which
generates a pair using corecursion.
\begin{lstlisting}
codata Product$\langle$A : Set, B : Set$\rangle$ : Set where
   Fst : Product $\rat$ A
   Snd : Product $\rat$ B
pair$\langle$A : Set, B : Set$\rangle$ (x:A, y:B) = corec Unit where
                                      { Fst u $\rat$ x
                                      ; Snd u $\rat$ y} @ $\lozenge$
\end{lstlisting}
For types with other contexts, we have to define different product types.  For
example, if $\mathtt{B}$\;depends on $\mathtt{Nat}$, we define the product like the following:
\begin{lstlisting}
codata Pair$\langle$A : Set, B : (n : Nat) $\rat$ Set$\rangle$ : (n : Nat) $\rat$ Set where
  First : (n : Nat) $\rat$ Pair n $\rat$ A
  Second : (n : Nat) $\rat$ Pair n $\rat$ B @ n
\end{lstlisting}
Here, the product also depends on $\mathtt{Nat}$. If $\mathtt{A}$\;or $\mathtt{B}$\;depends on values the
product must also depend on these values. This is the product, which is used
for the definition of vectors in \cite{basold2016type}.

On $\mathtt{Product}$\;we can define the swap function.
\begin{lstlisting}
swap$\langle$A : Set, B : Set$\rangle$ =
  corec Product$\langle$A,B$\rangle$ to Product$\langle$B,A$\rangle$ where
         Fst x $\rat$ Snd x
         Snd x $\rat$ Fst x
\end{lstlisting}
This is a well-typed function as shown by the following proof
\begin{scprooftree}{0.95}
\AxiomC{$(A : *, B : *)\mid\emptyset\mid\vdash$ Product$\langle$A,B$\rangle$ : $*$}
\AxiomC{$(A : *, B : *)\mid\emptyset\mid(x:A) \vdash$ Snd @ x : Product$\langle$A,B$\rangle$ \circled{a}}
 \noLine
 \UnaryInfC{$(A : *, B : *)\mid\emptyset\mid(y : B) \vdash$ Fst @ y : Product$\langle$A,B$\rangle$ \circled{b}}
 \BinaryInfC{$(A : *, B : *)\mid\emptyset\mid \vdash$ swap : (p : Product$\langle$A,B$\rangle$) $\rat$ Product$\langle$B,A$\rangle$}
 \end{scprooftree}
We show the derivation of \circled{a} in the following proof.  The
derivation of \circled{b} works analog.
\begin{prooftree}
\AxiomC{$(A : *, B : *)\mid\emptyset\mid(x : A) \vdash$ Snd : $(x : A) \rat$ Product$\langle$A,B$\rangle$}
\AxiomC{$(x : A )(x)\rightsquigarrow A$}
\UnaryInfC{$(x : A) \vdash x : A$}
\BinaryInfC{$(A : *, B : *)\mid\emptyset\mid(x : A) \vdash$ Snd @ x : Product$\langle$A,B$\rangle$}
\end{prooftree}
For the sake of brevity, we omitted the \(\beta\text{-equality}\) premises and the checking for of
the parameters. The \(\beta\text{-equality}\) premises wouldn't be interesting because
the involved terms are all already syntactically identical.

The Binary Coproduct corresponds to the $\mathtt{Either}$\;type in Haskell.  It is defined
as an inductive type.  We have one constructor $\mathtt{Left}$\;for $\mathtt{A}$\;and one
constructor Right for $\mathtt{B}$.
\begin{lstlisting}
data Coproduct$\langle$A,B$\rangle$ : Set where
   Left : A $\rat$ Coproduct
   Right : B $\rat$ Coproduct
\end{lstlisting}

\section{Sigma and Pi Type}
\label{sec:org68e71a8}

The \(\Sigma\text{-type}\) is a dependent pair of two types.  The second type can
depend on a value of the first type. It corresponds to the existential
quantifier in logic. We define it as an inductive type and call the
constructor $\mathtt{Exists}$.
\begin{lstlisting}
data Sigma$\langle$A : Set,B : (x : A) $\rat$ Set$\rangle$ : Set where
   Exists : (x:A) $\rat$ B x $\rat$ Sigma
\end{lstlisting}

The \(\Pi\text{-type}\) generalizes functions to dependent functions. The type of
the codomain or result of such a function can depend on the argument of the
function. We define the \(\Pi\text{-type}\) as a coinductive type. To destruct a
function we just apply it to a value. So the destructor is $\mathtt{Apply}$.

\begin{lstlisting}
codata Pi$\langle$A : Set,B : (x : A) $\rat$ Set$\rangle$ : Set where
   Apply : (x : A) $\rat$ Pi x $\rat$ B @ x
\end{lstlisting}

To construct a function we use corecursion on $\mathtt{Unit}$. The identity function
is defined like this
\begin{lstlisting}
id$\langle$A : Set$\rangle$ = corec Unit to Pi$\langle$A,(v:A).A$\rangle$ where
       { Apply v p = v } @ $\lozenge$
\end{lstlisting}

Evaluating the identity function on the number one results in the following
evaluation steps.

\begin{lstlisting}
apply = Apply$\langle$Nat,(v : Nat).Nat$\rangle$
one = S @ (Z @ )
apply @ id$\langle$Nat$\rangle$ @ one
= apply @ one @ ((corec Unit to Pi$\langle$Nat,(x:Nat).Nat$\rangle$ where
                    Apply v p = v ) @ $\lozenge$)
$\succ \widehat{\text{Nat}} \left(\underbrace{
   \begin{subarray}{c}
     \text{corec Unit to Pi where} \\
     \text{  \{Apply' v \_ = v\} @ x}
   \end{subarray}}_t\right)$[v/x][one,$\lozenge$]
= (rec Nat to Nat where
     Zero x = Zero @ ($\widehat{\text{Unit}}$(t,x))
     Succ x = Suc @ ($\widehat{Y}$(t,x)))@x[v/x][one,$\lozenge$]
= (rec Nat to Nat where
     Zero x = Zero @ ($\widehat{\text{Unit}}$(t))
     Succ x = Suc @ x)@x[v/x][one,$\lozenge$]
= (rec Nat to Nat where
     Zero x = Zero @ ($\widehat{\text{Unit}}$())
     Succ x = Suc @ x) @ x[v/x][one,$\lozenge$]
= (rec Nat to Nat where
     Zero x = Zero @ x
     Succ x = Suc @ x) @ x[v/x][one,$\lozenge$]
= (rec Nat to Nat where
     Zero x = Zero @ x
     Succ x = Suc @ x) @ v[one,$\lozenge$]
= (rec Nat to Nat where
     Zero x = Zero @ x
     Succ x = Suc @ x) @ one
= one
\end{lstlisting}

\section{Vectors and Streams}
\label{sec:orga00046d}

Vectors are a standard example for dependent types. They are like lists,
except that their type depends on their length. For example, a vector $\mathtt{[1;2]}$
has type $\mathtt{Vector\langle\;Nat\rangle\;2}$, because its length is 2. It has 2
constructors $\mathtt{Nil}$\;and $\mathtt{Cons}$\;like lists. $\mathtt{Nil}$\;gives back the empty vector
which has type $\mathtt{Vector\;0}$. The second constructor $\mathtt{Cons}$\;takes a natural
number $\mathtt{k}$, a value of type $\mathtt{A}$\;and a vector of length $\mathtt{k}$, a $\mathtt{Vector\;k}$. It
returns a new vector of type $\mathtt{Vector\;(Suc\;k)}$, whose head is the first
argument and its tail the second. In \cite{basold2016type} the head and tail
are encoded in a pair.

\begin{lstlisting}
data Vector$\langle$A : Set$\rangle$ : (n:Nat) $\rat$ Set where
  Nil : Unit $\rat$ Vector zero
  Cons : (k:Nat, v:A) $\rat$ Vector @ k $\rat$ Vector (Suc @ k)
nil$\langle$A : Set$\rangle$ = Nil$\langle$A : Set$\rangle$ @ $\lozenge$
\end{lstlisting}

The function $\mathtt{extend}$\;takes a value $\mathtt{x}$\;and extends it to a vector.
\begin{lstlisting}
extend$\langle$A : Set$\rangle$ =
  rec Vec$\langle$A$\rangle$ to ((x).Vec$\langle$ A$\rangle$ @ (Suc x) where
    Nil u = Cons$\langle$A$\rangle$ @ x @ nil$\langle$A$\rangle$
    Cons k v = Cons$\langle$A$\rangle$ @ (Suc @ k) @ v
\end{lstlisting}
The typechecking of this function goes as follows:
\begin{scprooftree}{0.75}
\AxiomC{(A : Set)$\mid\emptyset\mid\vdash$ (x).(Vec$\langle$A$\rangle$ @ (Suc @ x)) : (k: Nat) $\rat$ *}
\noLine
\UnaryInfC{(A: Set)$\mid\emptyset\mid$(u : A) $\vdash$ Cons$\langle$A$\rangle$ @ 0 @ (Nil$\langle$A$\langle$ @ ) : (x).(Vec$\langle$A$\rangle$ @ (Suc @ x)) @ 0}
\noLine
\UnaryInfC{(k : Nat, v : (x).(Vec @ (Suc @ x)) @ k) $\vdash$ Cons$\rangle$A$\langle$ @ (Suc @ k) @ v : (x).(Vec @ (Suc @ x)) @ (Suc @ k)}
\UnaryInfC{$\vdash$ extend$\langle$A$\rangle$ : (k:Nat,y : Vec$\langle$A$\rangle$ @ k) $\rat$ (x).(Vec$\langle$A$\rangle$ @ (Suc x)) @ k}
\end{scprooftree}
As an example, we evaluate a vector of length 1 with this function.  We choose length one
to see all $\mathtt{rec}$\;cases.
{\scriptsize
\begin{flalign*}
  &\extend\langle\Nat\rangle \apply 1 \apply (\Cons\langle\Nat\rangle \apply 0 \apply 0 \apply \nil\langle\Nat\rangle)&\\
  &= \extend\langle\Nat\rangle\apply(\Suc \apply k \bullet 0) \apply (\Cons\langle\Nat\rangle \apply 0 \apply 0 \apply \nil\langle\Nat\rangle)&\\
  &\succ \Cons\langle\Nat\rangle \apply (\Suc \apply k) \apply v \left[ \widehat{X\apply k}(\extend\langle\Nat\rangle \apply n \apply x)/v \right][0,\nil\langle\Nat\rangle]&\\
  &= \Cons\langle\Nat\rangle \apply (\Suc \apply k) \apply v \left[ \widehat{X}(\extend\langle\Nat\rangle \apply n \apply x)[k/n]/v \right][0,\nil\langle\Nat\rangle]&\\
  &= \Cons\langle\Nat\rangle \apply (\Suc \apply k) \apply v \left[ \extend \apply n \apply x[k/n]/v \right][0, \nil\langle\Nat\rangle]&\\
  &= \Cons\langle\Nat\rangle \apply (\Suc \apply k) \apply v \left[ \extend \apply k \apply x/v \right][0, \nil\langle\Nat\rangle]&\\
  &= \Cons\langle\Nat\rangle \apply (\Suc \apply k) \apply (\extend \apply k \apply x) [0,\nil\langle\Nat\rangle]&\\
  &= \Cons\langle\Nat\rangle \apply (\Suc \apply 0) \apply (\extend \apply 0 \apply (\nil\langle\Nat\rangle))&\\
  &= \Cons\langle\Nat\rangle \apply 1 \apply (\extend \apply 0 \apply (\Nil\langle\Nat\rangle \apply ))&\\
  &\succ \Cons\langle\Nat\rangle \apply 1 \apply (\Cons\langle\Nat\rangle \apply 0 \apply (\Nil\langle\Nat\rangle \apply ))\left[ \widehat{\Unit}(\extend \apply k \apply x) / u  \right][\lozenge]&\\
  &= \Cons\langle\Nat\rangle \apply 1 \apply (\Cons\langle\Nat\rangle \apply 0 \apply (\Nil\langle\Nat\rangle \apply x))[\lozenge]&\\
  &= \Cons\langle\Nat\rangle \apply 1 \apply (\Cons\langle\Nat\rangle \apply 0 \apply (\Nil\langle\Nat\rangle \apply ))&
\end{flalign*}}
Here, we write \(1\) for $\mathtt{Suc\;@\;(Zero\;@\;)}$\;and \(0\) for $\mathtt{Zero\;@\;\lozenge}$.

With the help of extended naturals, we can define partial streams. Those are
streams that depend on their definition depth. Like non-dependent streams,
they are coinductive and have 2 destructors for head and tail.
\begin{lstlisting}
codata PStr$\langle$A : Set$\rangle$: (n: ExNat) $\rat$ Set where
   hd : (k : ExNat) $\rat$ PStr$\langle$A$\rangle$ (succE k) $\rat$ A
   tl : (k : ExNat) $\rat$ PStr$\langle$A$\rangle$ (succE k) $\rat$ PStr$\langle$A$\rangle$ @ k
\end{lstlisting}
These streams are like vectors except they also can be infinite long. This is
in contrarst to non-dependent streams.  A non-dependent stream could not be of
length zero.  Because then a call of $\mathtt{hd}$\;and $\mathtt{tl}$\;on it wouldn't be defined.  In the
dependent case, the typechecker wouldn't allow such a call because $\mathtt{hd}$\;and
$\mathtt{tl}$\;expect streams which are at least of length one.  We can then define
$\mathtt{repeat}$.
\begin{lstlisting}
repeat$\langle$A : Set$\rangle$(x : A, n : Conat) =
  corec (n : Conat).Unit to PStr$\langle$A$\rangle$ where
    { Hd k s = x
    ; Tl k s = $\lozenge$ } @ n @ $\lozenge$
\end{lstlisting}
This function gets a value and an extended natural number. It generates a
constant partial stream of that value with the number as its length.


\chapter{Conclusion}
\label{sec:org82b5582}
We have implemented a dependent type theory with inductive and coinductive
types. In this theory, contrary to Coq and Agda, coinductive types can also
depend on values. Contrary to the theory of the paper we can define
parameterized types like $\mathtt{Maybe\langle\;A\;:\;Set\rangle}$\;where $\mathtt{A}$
can be an arbitrary type of kind $\mathtt{Set}$.

One downside is that we don't have universes. This prevents type polymorphism.
Further work needs to be done to solve this. Another problem is, that each
constructor or destructor has at least one argument. This restriction exists
in the original system to guarantee only strictly positive occurrences of
recursive arguments. For example, we have to apply a unit to the constructors
of a boolean type. We could allow recursive occurrences in the contexts of the
constructors and destructors. This makes it possible to remove the argument
with the recursive occurrence. Then we have to change the evaluation rules.

Our system allowed us to define the dependent function type. Therefore, we
need builtin dependent functions. We are hopeful that in the future we get a
more mainstream language, like Coq or Agda, where the dependent function type
is definable. As already mentioned in the introduction, this would lead to a
more symmetrical language.

\appendix

\chapter{Additional Proofs}
\label{sec:orgff507b7}

In this appendix we provide the missing proof for type applications.
\begin{lemma}
\((\Gamma).A@\id{\Gamma}\leftrightarrow_T A\)
\label{abstrid}
\end{lemma}
\begin{proof}
We show this by induction on the length of \(\Gamma\)
\begin{itemize}
\item \(\Gamma=\epsilon\):
\begin{equation*}
   A \longleftrightarrow_T A
\end{equation*}
\item \(\Gamma=x:B,\Gamma'\):
\begin{equation*}
  (x:B,\Gamma').A@x@\id{\Gamma'}
  \longrightarrow_p(\Gamma').A@\id{\Gamma'}[x/x]
  = (\Gamma').A@\id{\Gamma'} \overset{IdH.}{\longleftrightarrow_T}A
\end{equation*}
\end{itemize}
\end{proof}
\begin{lemma}
The following rule holds
\begin{prooftree}
\AxiomC{$x:A\vdash t:B$}
\AxiomC{$A\longleftrightarrow_TA'$}
\BinaryInfC{$x:A'\vdash t:B$}
\end{prooftree}
\label{ctxconv}
\end{lemma}
\begin{proof}
We show this by induction on t
\end{proof}
\begin{theorem}
The typing rule (5) in the paper holds
\begin{prooftree}
  \AxiomC{$X:\Gamma_1\rat*\mid\Gamma'\vdash C:\Gamma\rat*$}
  \AxiomC{$\Gamma_1,x:A\vdash t:B$}
  \BinaryInfC{$\Gamma',\Gamma,x:\widehat{C}(A)\vdash\widehat{C}(t):\widehat{C}(B) $}
\end{prooftree}
\end{theorem}
\begin{proof}
First we will generalize the rule to
\begin{prooftree}
  \AxiomC{$X_1:\Gamma_1\rat*,\dots,X_n:\Gamma_n\rat*\mid\Gamma'\vdash C:\Gamma\rat*$}
  \AxiomC{$\Gamma_i,x:A_i\vdash t_i:B_i$}
  \BinaryInfC{$\Gamma',\Gamma,x:\widehat{C}(\vv{A})\vdash\widehat{C}(\vv{t}):\widehat{C}(\vv{B}) $}
\end{prooftree}
Then, we gonna show this generealization by induction on the derivation \(\mathcal{D}\) of \(C\).
\begin{itemize}
\item $\mathcal{D}$ =
  \AxiomC{}
  \topI{$\vdash\top:*$}
  \DisplayProof

Then, the type actions is calculated as follows:
\begin{align*}
  &\widehat{\top}(\vv{A}) = \widehat{\top}() = \top\\
  &\widehat{\top}(\vv{t}) = \widehat{\top}() = x\\
  &\widehat{\top}(\vv{B}) = \widehat{\top}() = \top
\end{align*}
We than get the following derivation:
\begin{prooftree}
  \AxiomC{$\vdash\top:*$}
  \RightLabel{\textbf{(Proj)}}
  \UnaryInfC{$x:\top\vdash x:\top$}
\end{prooftree}
\item $\mathcal{D}$ =
  \Di{1}
  \UnaryInfC{$X_1:\Gamma_1\rat*,\dots,X_{n-1}:\Gamma_{n-1}$\TyCtx}
  \Di{2}
  \UnaryInfC{$\Gamma_n$\Ctx}
  \TyVarI{$X_1:\Gamma_1\rat*,\dots,X_n:\Gamma_n\rat*\mid\emptyset\vdash X_n:\Gamma_n\rat*$}
  \DisplayProof

Again, we calculate the type actions.
\begin{align*}
  &\widehat{X_n}(\vv{A}) = X_n[\vv{(\Gamma_i).A}/\vv{X}]@\id{\Gamma_n}= X_n[(\Gamma_n).A_n/X_n]@\id{\Gamma_n} = (\Gamma_n).A_n@\id{\Gamma_n}\\
  &\widehat{X_n}(\vv{t}) = t_n\\
  &\widehat{X_n}(\vv{B}) = X_n[\vv{(\Gamma_i).B}/\vv{X}]@\id{\Gamma_n}= X_n[(\Gamma_n).B_n/X_n]@\id{\Gamma_n} = (\Gamma_n).B_n@\id{\Gamma_n}\\
\end{align*}
We know from the first premise that \(\Gamma=\Gamma_n\) and \(\Gamma'=\emptyset\).

Here, we get the derivation:
\begin{scprooftree}{0.9}
\AxiomC{$\Gamma_n,x:A\vdash t:B$}
\AxiomC{}
\RightLabel{Thrm. \ref{abstrid}}
\UnaryInfC{$A\longleftrightarrow_T(\Gamma_n).A@\id{\Gamma_n}$}
\RightLabel{Thrm. \ref{ctxconv}}
\BinaryInfC{$\Gamma_n,x:(\Gamma_n).A@\id{\Gamma_n}\vdash t:B$}
\AxiomC{}
\RightLabel{Thrm. \ref{abstrid}}
\UnaryInfC{$B\longleftrightarrow_T(\Gamma_n).B@\id{\Gamma_n}$}
\RightLabel{Conv}
\BinaryInfC{$\Gamma_n,x:(\Gamma_n).A@\id{\Gamma_n}\vdash t_n:(\Gamma_n).B@\id{\Gamma_1}$}
\end{scprooftree}

\item $\mathcal{D}$ =
  \Di{1}
  \UnaryInfC{$X_1:\Gamma_1\rat*,\dots,X_n:\Gamma_n\mid\Gamma'\vdash C:\Gamma\rat*$}
  \Di{2}
  \UnaryInfC{$\Gamma_n$\Ctx}
  \TyVarWeak{$X_1:\Gamma_1\rat*,\dots,X_{n+1}:\Gamma_{n+1}\rat*\mid\Gamma'\vdash C:\Gamma\rat*$}
  \DisplayProof

Here, we get the derivation:
\begin{prooftree}
  \AxiomC{$X_1:\Gamma_1\rat*,\dots,X_{n+1}:\Gamma_{n+1}\rat*\mid\Gamma'\vdash C:\Gamma\rat*$}
  \RightLabel{(*)}
  \UnaryInfC{$X_1:\Gamma_1\rat*,\dots,X_n:\Gamma_n\rat*\mid\Gamma'\vdash C:\Gamma\rat*$}
  \AxiomC{$\Gamma_i,x:A_i\vdash t_i:B_i$}
  \RightLabel{IdH.}
  \BinaryInfC{$\Gamma',\Gamma,x:\underbrace{\widehat{C}(\vv{A})}_{\overset{(**)}{=}\widehat{C}(\vv{A},A_{n+1})}\vdash\underbrace{\widehat{C}(\vv{t})}_{\overset{(***)}{=}\widehat{C}(\vv{t},t_{n+1})}:\underbrace{\widehat{C}(\vv{B})}_{\overset{(**)}{=}\widehat{C}(\vv{B},B_{n+1})} $}
\end{prooftree}

(\texttt{*}) Here, we undo \textbf{(TyVar-Weak)}

(\texttt{**}) \(X_{n+1}\) doesn't occur free in C, otherwise \(\mathcal{D}_1\) wouldn't be possible

(\texttt{***}) Case for \textbf{(TyVar-Weak)} of type actions on terms

\item $\mathcal{D}$ =
  \begin{scprooftree}{0.8}
  \Di{1}
  \UnaryInfC{$X_1:\Gamma_1\rat*,\dots,X_n:\Gamma_n\mid\Gamma'\vdash C:\Gamma\rat*$}
  \Di{2}
  \UnaryInfC{$X_1:\Gamma_1\rat*,\dots,X_n:\Gamma_n\mid\Gamma'\vdash D:*$}
  \TyWeak{$X_1:\Gamma_1\rat*,\dots,X_n:\Gamma_n\rat*\mid\Gamma',y:D\vdash C:\Gamma\rat*$}
  \end{scprooftree}

Here, we get the derivation:
\begin{scprooftree}{0.6}
  \AxiomC{$X_1:\Gamma_1\rat*,\dots,X_n:\Gamma_n\rat*\mid\Gamma',y:D\vdash C:\Gamma\rat*$}
  \RightLabel{(*)}
  \UnaryInfC{$X_1:\Gamma_1\rat*,\dots,X_n:\Gamma_n\rat*\mid\Gamma'\vdash C:\Gamma\rat*$}
  \AxiomC{$\Gamma_i,x:A_i\vdash t_i:B_i$}
  \RightLabel{IdH.}
  \BinaryInfC{$\Gamma',\Gamma,x:\widehat{C}(\vv{A})\vdash\widehat{C}(\vv{t}):\widehat{C}(\vv{B})$}
  \AxiomC{$X_1:\Gamma_1\rat*,\dots,X_n:\Gamma_n\mid\Gamma'\vdash D:*$}
  \TermWeak{$\Gamma',\Gamma,x:\widehat{C}(\vv{A})y\vdash\widehat{C}(\vv{t}):\widehat{C}(\vv{B})$}
\end{scprooftree}

(\texttt{*}) Here, we undo \textbf{(Ty-Weak)}

\item $\mathcal{D}= $
\AxiomC{$X_1:\Gamma_1,\ldots,X_n:\Gamma_n\mid\Gamma'\vdash C':(y:D,\Gamma)\rat* $}
\AxiomC{$\Gamma'\vdash s: D$}
\TyInst{$X_1:\Gamma_1,\ldots,X_n:\Gamma_n\mid\Gamma'\vdash C'@s:\Gamma\rat* $}
\DisplayProof

Then, we get the following induction hypothesis:
\begin{prooftree}
  \AxiomC{$X_1:\Gamma_1\rat*,\dots,X_n:\Gamma_n\rat*\mid\Gamma'\vdash C':(y:D,\Gamma)\rat*$}
  \AxiomC{$\Gamma_i,x:A_i\vdash t_i:B_i$}
  \BinaryInfC{$\Gamma',y:D,\Gamma,x:\widehat{C'}(\vv{A})\vdash\widehat{C'}(\vv{t}):\widehat{C'}(\vv{B}) $}
\end{prooftree}

The calculated type actions are:
\begin{align*}
  &\widehat{C'@s}(\vv{A})=C'@s[\vv{(\Gamma_i).A}/\vv{X}]@\id{\Gamma}=C'[\vv{(\Gamma_i).A}/\vv{X}]@s@\id{\Gamma}
  =\widehat{C'}(\vv{A})[s/y]\\
  &\widehat{C'@s}(\vv{t})=\widehat{C'}(\vv{t})[s/y]\\
  &\widehat{C'@s}(\vv{B})=C'@s[\vv{(\Gamma_i).B}/\vv{X}]@\id{\Gamma}=C'[\vv{(\Gamma_i).B}/\vv{X}]@s@\id{\Gamma}
  =\widehat{C'}(\vv{B})[s/y]\\
\end{align*}

We then get the following derivation:
\begin{scprooftree}{0.95}
  \AxiomC{$X_1:\Gamma_1\rat*,\dots,X_n\rat*\mid\Gamma_2'\vdash C'@s:\Gamma_2[s/y]\rat*$}
  \RightLabel{(*)}
  \UnaryInfC{$X_1:\Gamma_1\rat*,\dots,X_n:\Gamma_n\rat*\mid\Gamma_2'\vdash C':(y:D,\Gamma_2)\rat*$}
  \AxiomC{$\Gamma_i,x:A_i\vdash t_i:B_i$}
  \RightLabel{IdH.}
  \BinaryInfC{$\Gamma_2',y:D,\Gamma_2,x:\widehat{C'}(\vv{A})\vdash\widehat{C'}(\vv{t}):\widehat{C'}(\vv{B}) $}
  \UnaryInfC{$\Gamma_2',\Gamma_2[s/y],x:\widehat{C'}(\vv{A})[s/y]\vdash\widehat{C'}(\vv{t})[s/y]:\widehat{C'}(\vv{B})[s/y] $}
\end{scprooftree}
(\texttt{*}) This is the reverse of \textbf{(Ty-Inst)}.

\item $\mathcal{D}= $
\AxiomC{$X_1:\Gamma_1,\ldots,X_n:\Gamma_n\mid\Gamma',y:D\vdash C':\Gamma\rat* $}
\ParamAbstr{$X_1:\Gamma_1,\ldots,X_n:\Gamma_n\mid\Gamma'\vdash (y).C':(y:D,\Gamma)\rat* $}
\DisplayProof

The calculated type actions are:
\begin{align*}
  \widehat{(y).C'}(\vv{A})&=(y).C'[\vv{(\Gamma_i.A)}/\vv{X}]@\id{\Gamma}\\
                     &=(y).(C'[\vv{(\Gamma_i.A)}/\vv{X}])@y@\id{\Gamma}\\
                     &\longleftrightarrow_T(C'[\vv{(\Gamma_i.A)}/\vv{X}])@\id{\Gamma}\\
                     &=\widehat{C'}(\vv{A})\\
  \widehat{(y).C'}(\vv{t})&=\widehat{C'}(\vv{t})\\
  \widehat{(y).C'}(\vv{B})&=(y).C'[\vv{(\Gamma_i.B)}/\vv{X}]@\id{\Gamma}\\
                     &=(y).(C'[\vv{(\Gamma_i.B)}/\vv{X}])@y@\id{\Gamma}\\
                     &\longleftrightarrow_T(C'[\vv{(\Gamma_i.B)}/\vv{X}])@\id{\Gamma}\\
                     &=\widehat{C'}(\vv{B})\\
\end{align*}

The derivation then becomes the following:
\begin{scprooftree}{0.95}
  \AxiomC{$X_1:\Gamma_1\rat*,\dots,X_n:\Gamma_n\rat*\mid\Gamma'\vdash (y).C':(y:D,\Gamma)\rat*$}
  \RightLabel{(*)}
  \UnaryInfC{$X_1:\Gamma_1\rat*,\dots,X_n:\Gamma_n\rat*\mid y:D,\Gamma'\vdash C':\Gamma\rat*$}
  \AxiomC{$\Gamma_i,x:A_i\vdash t_i:B_i$}
  \RightLabel{IdH.}
  \BinaryInfC{$y:D,\Gamma',\Gamma,x:\widehat{C'}(\vv{A})\vdash\widehat{C'}(\vv{t}):\widehat{C'}(\vv{B})$}
\end{scprooftree}
(\texttt{*}) This is the reverse of \textbf{(Param-Abstr)}.

\item \(\mathcal{D}\) =
\begin{prooftree}
    \Di{1}
    \UnaryInfC{$\sigma_k:\Delta_k\triangleright\Gamma$}
    \Di{2}
    \UnaryInfC{$X_1:\Gamma_1\rat*,\dots,X_n\rat*,X:\Gamma\rat*\mid\Delta_k\vdash D_k:*$}
    \FPTy
    \BinaryInfC{$X_1:\Gamma_1\rat*,\dots,X_{n}\rat*\mid\emptyset\vdash\mu(Y:\Gamma\rat*;\vv{\sigma};\vv{D}):\Gamma\rat*$}
\end{prooftree}
From this we know \(\Gamma'=\emptyset\)

The calculated type actions are:
\begin{align*}
  &\widehat{\mu(Y:\Gamma\rat*;\vv{\sigma};\vv{D})}(\vv{A})\\
  &=\mu(Y:\Gamma\rat*;\vv{\sigma};\vv{D})[\vv{(\Gamma_i).A}/\vv{X}]@\id{\Gamma}\\
  &=\mu(Y:\Gamma\rat*;\vv{\sigma};\vv{D}[\vv{(\Gamma_i).A}/\vv{X}])@\id{\Gamma}\\
  &\widehat{\mu(Y:\Gamma\rat*;\vv{\sigma};\vv{D})}(\vv{t})\\
  &=\text{rec}^{\mu(Y:\Gamma\rat*;\vv{\sigma};\vv{D}[\vv{(\Gamma_i).A}/\vv{X}])}\vv{(\Delta_k,x).\alpha_k@\id{\Delta_k}@\widehat{D_k}(\vv{t},x)}@\id{\Gamma}@x\\
  &\widehat{\mu(Y:\Gamma\rat*;\vv{\sigma};\vv{D})}(\vv{B})\\
  &=\mu(Y:\Gamma\rat*;\vv{\sigma};\vv{D})[\vv{(\Gamma_i).B}/\vv{X}]@\id{\Gamma}\\
  &=\mu(Y:\Gamma\rat*;\vv{\sigma};\vv{D}[\vv{(\Gamma_i).B}/\vv{X}])@\id{\Gamma}
\end{align*}

From the assumptions
\begin{align*}
&X_1:\Gamma_1\rat*,\dots,X_n:\Gamma_n\rat*\mid\emptyset\vdash \mu(Y:\Gamma\rat*;\vv{\sigma};\vv{D}):\Gamma\rat*\\
&\Gamma_i,x:A_i\vdash t_i:B_i
\end{align*}
we have to proof that in the context
\begin{equation*}
 \Gamma,x:\mu(Y:\Gamma\rat*;\vv{\sigma};\vv{D}[\vv{(\Gamma_i).A}/\vv{B}])@\id{\Gamma}
\end{equation*}
the expression
\begin{equation*}
 \text{rec}^{\mu(Y:\Gamma\rat*;\vv{\sigma};\vv{D}[\vv{(\Gamma_i).A}/\vv{X}])}\vv{(\Delta_k,y).\alpha_k@\id{\Delta_k}@\widehat{D_k}(t,y)}@\id{\Gamma}@x
\end{equation*}
has type
\begin{equation*}
\mu(Y:\Gamma\rat*;\vv{\sigma};\vv{D}[\vv{(\Gamma_i).B}/\vv{X}])@\id{\Gamma}
\end{equation*}
We can use the induction hypothesis
\begin{prooftree}
  \AxiomC{$X_1:\Gamma_1\rat*,\dots,X_n:\Gamma_n\rat*,Y:\Gamma_{n+1}\rat*\mid\Delta_k\vdash D_k:*$}
  \AxiomC{$\Gamma_i,x:A_i\vdash t_i:B_i$}
  \BinaryInfC{$\Delta_k,x:\widehat{D_k}(\vv{A},A_{n+1})\vdash\widehat{D_k}(\vv{t},y):\widehat{D_k}(\vv{B},B_{n+1}) $}
\end{prooftree}
See \ref{sec:prc} for a proof of it.

\item \(\mathcal{D}\) =
\begin{prooftree}
    \Di{1}
    \UnaryInfC{$\sigma_k:\Delta_k\triangleright\Gamma$}
    \Di{2}
    \UnaryInfC{$X_1:\Gamma_1\rat*,\dots,X_n\rat*,X:\Gamma\rat*\mid\Delta_k\vdash D_k:*$}
    \FPTy
    \BinaryInfC{$X_1:\Gamma_1\rat*,\dots,X_{n}\rat*\mid\emptyset\vdash\nu(Y:\Gamma\rat*;\vv{\sigma};\vv{D}):\Gamma\rat*$}
\end{prooftree}
From this we know \(\Gamma'=\emptyset\).

The calculated type actions are:
\begin{align*}
  &\widehat{\nu(Y:\Gamma\rat*;\vv{\sigma};\vv{D})}(\vv{A})\\
  &=\nu(Y:\Gamma\rat*;\vv{\sigma};\vv{D})[\vv{(\Gamma_i).A}/\vv{X}]@\id{\Gamma}\\
  &=\nu(Y:\Gamma\rat*;\vv{\sigma};\vv{D}[\vv{(\Gamma_i).A}/\vv{X}])@\id{\Gamma}\\
  &\widehat{\nu(Y:\Gamma\rat*;\vv{\sigma};\vv{D})}(\vv{t})\\
  &=\text{corec}^{\nu(Y:\Gamma\rat*;\vv{\sigma};\vv{D}[(\vv{\Gamma_i).B}/\vv{X}])}\vv{(\Delta_k,x)\widehat{D_k}(\vv{t},x)[(\xi_k@\id{\Delta_k}@x)/x]}@\id{\Gamma}@x\\
  &\widehat{\nu(Y:\Gamma\rat*;\vv{\sigma};\vv{D})}(\vv{B})\\
  &=\nu(Y:\Gamma\rat*;\vv{\sigma};\vv{D})[\vv{(\Gamma_i).B}/\vv{X}]@\id{\Gamma}\\
  &=\nu(Y:\Gamma\rat*;\vv{\sigma};\vv{D}[\vv{(\Gamma_i).B}/\vv{X}])@\id{\Gamma}
\end{align*}

From the assumptions
\begin{align*}
&X_1:\Gamma_1\rat*,\dots,X_n:\Gamma_n\rat*\mid\emptyset\vdash \nu(Y:\Gamma\rat*;\vv{\sigma};\vv{D}):\Gamma\rat*\\
&\Gamma_i,x:A_i\vdash t_i:B_i
\end{align*}
we have to proof that in the context
\begin{equation*}
 \Gamma,x:\nu(Y:\Gamma\rat*;\vv{\sigma};\vv{D}[(\Gamma_1).A/X])@\id{\Gamma}
\end{equation*}
the expression
\begin{equation*}
 \text{corec}^{\nu(Y:\Gamma\rat*;\vv{\sigma};\vv{D}[(\vv{\Gamma_i).B}/\vv{X}])}\vv{(\Delta_k,x)\widehat{D_k}(\vv{t},x)[(\xi_k@\id{\Delta_k}@x)/x]}@\id{\Gamma}@x\\
\end{equation*}
has type
\begin{equation*}
\nu(Y:\Gamma\rat*;\vv{\sigma};\vv{D}[\vv{(\Gamma_i).B}/\vv{X}])@\id{\Gamma}
\end{equation*}
We can use the induction hypothesis
\begin{prooftree}
  \AxiomC{$X_1:\Gamma_1\rat*,\dots,X_n:\Gamma_n\rat*,Y:\Gamma_{n+1}\rat*\mid\Delta_k\vdash D_k:*$}
  \AxiomC{$\Gamma_i,y_k:A_i\vdash t_i:B_i$}
  \BinaryInfC{$\Delta_k,y_k:\widehat{D_k}(\vv{A},A_{n+1})\vdash\widehat{D_k}(\vv{t},y):\widehat{D_k}(\vv{B},B_{n+1}) $}
\end{prooftree}
See \ref{sec:prc} for this proof.
\end{itemize}
\end{proof}

  \begin{landscape}
  \section{Proofs for Recursion and Corecursion}
  \label{sec:prc}
\begin{prooftree}
   \AxiomC{$\Gamma_1\vdash\sigma:\Gamma_2$}
   \AxiomC{$\Gamma_3\vdash\tau:\Gamma_1$}
   \RightLabel{($*$)}
   \BinaryInfC{$\Gamma_3\vdash\sigma\circ\tau:\Gamma_2$}
\end{prooftree}
      \begin{prooftree}
      \D
      \UnaryInfC{$\Delta,\Gamma_k,y_k:A_k[C/X]\vdash g_k:C@\sigma_k$}
      \RightLabel{TyAct}
      \UnaryInfC{$\Delta,\Gamma_k,x:A_k[\mu/X]\vdash g_k[\widehat{A_k}(\rec^\mu\vv{(\Gamma_k,y_k).g_k}@\id{\Gamma}@x/y_k]$}
      \D
      \UnaryInfC{$\Delta\vdash\tau:\Gamma_k$}
      \D
      \UnaryInfC{$\Delta\vdash u:A_k[\mu/X]$}
      \TrinaryInfC{$\Delta\vdash g_k[\widehat{A_k}(\rec^\mu\vv{(\Gamma_k,y_k).g_k}@\id{\Gamma}@x)/y_k][\tau,u]:C@\sigma_k$}
    \end{prooftree}
  \begin{scprooftree}{0.7}
    \D
    \UnaryInfC{$\Delta,\Gamma_k,y_k:\Delta_k[C/X]\vdash g_k: C@\sigma_k$}
    \IndE
    \UnaryInfC{$\Delta\vdash\rec^\mu\vv{(\Gamma_k,y_k).g_k}:(\Gamma,x:\mu@\sigma_k)\rat C@\sigma_k$}
    \D
    \UnaryInfC{$\Gamma_k\vdash\sigma_k:\Gamma$}
    \D
    \UnaryInfC{$\Delta\vdash\tau:\Gamma_k$}
    \RightLabel{($*$)}
    \BinaryInfC{$\Delta\vdash\sigma_k\circ\tau:\Gamma$}
    \Inst{$\Delta\vdash(rec^\mu\vv{(\Gamma_k,y_k).g_k}@(\sigma_k\circ\tau)):(x:\mu@\sigma_k)\rat C@\sigma_k$}
    \AxiomC{}
    \IndI{$\Delta\vdash\alpha_k^\mu:(\Gamma_k,y:A_k[\mu/X])\rat\mu@\sigma_k$}
    \D
    \UnaryInfC{$\Delta\vdash\tau:\Gamma_k$}
    \Inst{$\Delta\vdash\alpha_k^\mu @\tau: (y:A_k[\mu/X])\rat\mu@\sigma_k$}
    \D
    \UnaryInfC{$\Delta\vdash u:A_k[\mu/X])$}
    \Inst{$\Delta\vdash\alpha_k^\mu @\tau@u:\mu@\sigma_k$}
    \Inst{$\Delta\vdash(\rec^\mu\vv{(\Gamma_k,y_k).g_k}@(\sigma_k\circ\tau))@(\alpha_k^\mu @\tau@u):C@\sigma_k$}
  \end{scprooftree}
  \begin{scprooftree}{0.48}
    \AxiomC{$\vdash\overbrace{\mu[\vv{(\Gamma_k).A_k}/\vv{X_k}]}^C:\Gamma\rat*$}
    \AxiomC{}
    \IndI{$\Gamma,x:\mu[\vv{(\Gamma_k).A_k}/\vv{X_k}]@\id{\Gamma},\Delta_k,y_k:A_k[\mu[\vv{(\Gamma_k).B_k}/\vv{X_k}]/X]\vdash\alpha_k^{\mu[\vv{(\Gamma_k).B_k}/\vv{X_k}]}:(\Delta_k,y_k:A_k[\vv{(\Gamma_k).B_k}/\vv{X_k}])\rat\mu[\vv{(\Gamma_k).B_k}/\vv{X_k}]@\sigma_k$}
    \AxiomC{$\Gamma,x:\mu[\vv{(\Gamma_k).A_k}/\vv{X_k}]@\id{\Gamma},\Delta_k,y_k:\overbrace{A_k[\mu[\vv{(\Gamma_k).B_k}/\vv{X_k}]/X]}^{=\widehat{A_k}(\vv{B},?)}\vdash\widehat{A_k}(\vv{t},y_k):\overbrace{A_k[\mu[\vv{(\Gamma_k).B_k}/\vv{X_k}]/X]}^{=\widehat{A_k}(\vv{B},?)}$}
    \Inst{$\Gamma,x:\mu[\vv{(\Gamma_k).A_k}/\vv{X_k}]@\id{\Gamma},\Delta_k,y_k:A_k[\mu[\vv{(\Gamma_k).B_k}/\vv{X_k}]/X]\vdash\alpha_k^{\mu[\vv{(\Gamma_k).B_k}/\vv{X_k}]}@\id{\Delta_k}@\widehat{A_k}(\vv{t},y_k):\mu[\vv{(\Gamma_k).B_k}/\vv{X_k}]@\sigma_k$}
    \IndE
    \BinaryInfC{$\Gamma,x:\mu[\vv{(\Gamma_k).A_k}/\vv{X_k}]@\id{\Gamma}\vdash\rec^{\mu[\vv{(\Gamma_k).A_k}/\vv{X_k}]}\vv{(\Delta_k,y_k).\alpha_k^{\mu[\vv{(\Gamma_k).B_k}/\vv{X_k}]}@\id{\Delta_k}@\widehat{A_k}(\vv{t},y_k)}:(\Gamma,x:\mu[\vv{(\Gamma_k).A_k}/\vv{X_k}]@\id{\Gamma})\rat\mu[\vv{(\Gamma_k).B_k}/\vv{X_k}]@\id{\Gamma}$}
    \AxiomC{$\ldots$}
    \Inst{$\Gamma,x:\mu[\vv{(\Gamma_k).A_k}/\vv{X_k}]@\id{\Gamma}\vdash\rec^{\mu[\vv{(\Gamma_k).A_k}/\vv{X_k}]}\vv{(\Delta_k,y_k).\alpha_k^{\mu[\vv{(\Gamma_k).B_k}/\vv{X_k}]}@\id{\Delta_k}@\widehat{A_k}(\vv{t},y_k)}@\id{\Gamma}@x:\mu[\vv{(\Gamma_k).B_k}/\vv{X_k}]@\id{\Gamma}$}
  \end{scprooftree}
  \begin{scprooftree}{0.78}
    \AxiomC{$\vdash\overbrace{\nu[\vv{(\Gamma_k).A_k}/\vv{X_k}]}^C:\Gamma\rat*$}
    \AxiomC{$\Gamma,x:\nu[\vv{(\Gamma_k).A_k}/\vv{X_k}]@\id{\Gamma},\Delta_k,y_k:\nu[\vv{(\Gamma_k).A_k}/\vv{X_k}]@\sigma_k\vdash\widehat{A_k}(\vv{t},y_k)[(\xi_k^{\nu[\vv{(\Gamma_k).A_k}/\vv{X_k}]}@\id{\Delta_k}@y_k)/y_k]:A_k[\nu[\vv{(\Gamma_k).A_k}/\vv{X_k}]/X]$}
    \IndE
    \BinaryInfC{$\Gamma,x:\nu[\vv{(\Gamma_k).A_k}/\vv{X_k}]@\id{\Gamma}\vdash\corec^{\nu[\vv{(\Gamma_k).B_k}/\vv{X_k}]}\vv{(\Delta_k,y_k).\widehat{A_k}(\vv{t},y_k)[(\xi_k^{\nu[\vv{(\Gamma_k).A_k}/\vv{X_k}]}@\id{\Delta_k}@y_k)/y_k]}:(\Gamma,x:\nu[\vv{(\Gamma_k).A_k}/\vv{X_k}]@\id{\Gamma})\rat\nu[\vv{(\Gamma_k).B_k}/\vv{X_k}]$}
    \AxiomC{$\ldots$}
    \Inst{$\Gamma,x:\nu[\vv{(\Gamma_k).A_k}/\vv{X_k}]@\id{\Gamma}\vdash\corec^{\nu[\vv{(\Gamma_k).B_k}/\vv{X_k}]}\vv{(\Delta_k,y_k).\widehat{A_k}(\vv{t},y_k)[(\xi_k^{\nu[\vv{(\Gamma_k).A_k}/\vv{X_k}]}@\id{\Delta_k}@y_k)/y_k]}@\id{\Gamma}@x:\nu[\vv{(\Gamma_k).B_k}/\vv{X_k}]$}
  \end{scprooftree}
 \end{landscape}


\bibliographystyle{alpha}
\bibliography{references}
\end{document}
